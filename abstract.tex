Individuals commonly engage in collaborative behavior to more easily produce works of high societal impact. The effect of many individual characteristics such as age or gender on the effectiveness of a team is still unclear. Gender is especially pertinent because many professional settings are still far from gender parity, despite ongoing controversy about innate differences between males and females.

In this dissertation, I use a rigorous mathematical approach validated with large datasets to study the effect of gender diversity in scientific collaborations and movie productions, and the impact of scientific works.

First, I analyze the publication records of thousand of researchers in science, technology, engineering, and math disciplines and show that previous contradicting findings of gender differences in collaboration patterns are a by-product of females' historic disadvantages in academia. I also present evidence of gender segregation in some sub-disciplines of molecular biology.

While there have been claims that males may be better suited for research than females, the same cannot be said for the movie industry. Therefore, to ensure the generality of my findings, I also study gender diversity in U.S. movie casts. I demonstrate that a period of concentration of power at the hands of a small group of male leaders had a severe negative influence on female representation in the U.S. movie industry. Moreover, I find gender diversity among movie producers, directors, and actors to be strongly interdependent which can exacerbate female under-representation in movie casts.

The success of creative teams is also determined by how their work is received by their peers. Having limited time and expertise, individuals use a variety of measures to identify which books to read, movies to watch, songs to listen, or sights to see. Yet, most metrics are subjective measures of quality that can have unknown biases. I develop a principled indicator that quantifies the long-term impact of scientific works. By virtue of its construction, my indicator is resistant to manipulation and rewards publication quality over quantity.
