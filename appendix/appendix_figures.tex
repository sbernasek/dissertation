\chapter{Supplementary Figures}
\label{ch:sup_figures}

\begin{figure}[t]
\includegraphics[width=\textwidth]{figures/collaboration_gini}
\caption[Distribution of the Gini coefficient of collaboration.]{\textbf{Gender differences in the propensity to repeat previous collaboration measured using the Gini coefficient}. Distribution of the Gini coefficient of collaboration heterogeneity \cite{Ceriani2011} for females (orange) and males (purple) in the dataset with at least $10$ publications. We exclude single-author publications. We obtain $p$-values for the validity of the null hypothesis that the samples were drawn from the same distribution using the Kolmogorov-Smirnov test. For all disciplines, we find $\delta=2(\bar{G}_F-\bar{G}_M)/(\bar{G}_F+\bar{G}_M)<0$, where $\bar{G}_F$ and $\bar{G}_M$ are the average Gini coefficient of the female and male faculty, respectively. Females have Gini coefficients smaller than those of males, suggesting that female faculty have a lower propensity than male faculty to repeat collaborations.}
\label{fig:collaboration:gini}
\end{figure}

\clearpage

\begin{figure}[t]
\includegraphics[width=\textwidth]{figures/collaboration_disparity}
\caption[Distribution of the disparity index for repeating co-authors.]{\textbf{Gender difference in the propensity to repeat previous co-authors measured using the disparity index}. Distribution of the disparity index measuring the repetition of co-authors of females (orange) and males (purple). The $p$-values indicate the significance of the gender difference, obtained with Kolmogorov-Smirnov test. The result is in good agreement with that obtained using the Gini coefficient in Fig.~\ref{fig:collaboration:gini}.}
\label{fig:collaboration:disparity}
\end{figure}

\clearpage

\begin{figure}[t]
\includegraphics[width=\textwidth]{figures/collaboration_gini_cor}
\caption[Correlation between Gini coefficient and probability to repeat co-authors.]{\textbf{Correlation between Gini coefficient and probability to repeat previous co-authors}. Orange (female) and purple (male) lines are linear fits to data, and $R^2_F$ and $R^2_M$ are the corresponding coefficient of determination.}
\label{fig:collaboration:gini_cor}
\end{figure}

\begin{figure}[t]
    \begin{subfigure}{\textwidth}
    \begin{center}
        \includegraphics[scale=0.73]{figures/collaboration_sim_coau}
        \phantomcaption
        \label{fig:collaboration:sim_coau_a}
    \end{center}
    \end{subfigure}
    \begin{subfigure}{0\textwidth}
        \phantomcaption
        \label{fig:collaboration:sim_coau_b}
    \end{subfigure}
    \begin{subfigure}{0\textwidth}
        \phantomcaption
        \label{fig:collaboration:sim_coau_c}
    \end{subfigure}
\caption[Survival curves of the simulated total number of distinct co-authors.]{\textbf{Heterogeneity in the number of publications and team size masks the effect of gender difference in the propensity to repeat co-authors}. Survival curves of the simulated total number of distinct co-authors with fixed number of publications and team size (\textbf{A}), fixed number of publications and team sizes sampled from real data (\textbf{B}), and both number of publications and team sizes from real data (\textbf{C}) for female (orange) and male (purple) faculty in all departments (see Appendix \ref{sec:methods:collaboration} for details). We obtained $p$-values for the validity of the null hypothesis that the samples were drawn from the same distribution using the Kolmogorov-Smirnov test. Statistical significant results with $p < \mathrm{0.01}/\mathrm{18} \approx \mathrm{0.0006}$ (Bonferroni correction for multiple hypothesis) are shaded grey. When using fixed number of publications and team size, females have significantly more distinct co-authors. However, the gender difference disappears for most disciplines when using fixed number of publications but real team sizes. When we also use number of publications from the real data, females have significantly fewer distinct co-authors, consistent with Fig.~\ref{fig:collaboration:coau}.}
\label{fig:collaboration:sim_coau}
\end{figure}

\clearpage

\begin{figure}[t]
\includegraphics[width=\textwidth]{figures/collaboration_coau_growth}
\caption[Growth of average number of co-authors during considered period.]{\textbf{Growth of average number of co-authors during considered period}. Average number of co-authors per publication for females (orange) and males (purple) as a function of publication year. The data are smoothed using a moving averaging method with window size 3. The shaded region indicates the 99\% confidence interval obtained with bootstrapping.}
\label{fig:collaboration:coau_growth}
\end{figure}

\clearpage

\begin{figure}[t]
\includegraphics[width=\textwidth]{figures/collaboration_ratio_pub}
\caption[Ratio of number of publications authored by females.]{\textbf{In molecular biology departments, female faculty work in smaller teams than male faculty}. Logarithm of the ratio of observed number of publications authored by females over that expected from a hypergeometric distribution (orange circles). The publications are binned by the number of co-authors corrected for the annual average with a bin size of $0.2$. The shaded areas indicate that the observed number is significantly different from expected by the model, using the Bonferroni correction by treating each bin as an independent hypothesis test (see Appendix \ref{sec:methods:collaboration} for details). The error bars indicate thrice the standard deviation. The black line indicates the ratio of $1.0$, and the purple line indicates the average corrected team size. Note that for molecular biology, females have more publications than expected with smaller teams (corrected team size $<1.0$) and fewer publications than expected with larger teams (corrected team size $>1.0$).}
\label{fig:collaboration:ratio_pub}
\end{figure}

\clearpage

\begin{figure}[t]
\includegraphics[width=\textwidth]{figures/collaboration_pub_coau_cheme}
\caption[Average co-authors and publications in chemical engineering.]{\textbf{Correlation between the average number of co-authors corrected for the annual average versus the fraction of publications authored by female faculty in chemical engineering departments.} Publications are grouped by journal. We restricted the publication types to ``article'', ``letter'', and ``note''. The size of the circle is proportional to the logarithm of the number of publications in that journal or sub-discipline. We use journal category in the \textit{ISI Journal Citation Report} as the sub-disciplines. Journals with multiple categories are plotted as concentric rings. The purple line indicates the total average fraction of publications by females for all the publications authored by faculty in chemical engineering in our cohort, $f_M$. The blue line is a weighted linear regression, in which we assign to each journal a weight equal to the number of publications. We only include data points within the range of $[0.5f_M,\;2f_M]$.}
\label{fig:collaboration:pub_coau_cheme}
\end{figure}

\clearpage

\begin{figure}[t]
\includegraphics[width=\textwidth]{figures/collaboration_pub_coau_ch}
\caption[Average co-authors and publications in chemistry.]{\textbf{Correlation between the average number of co-authors corrected for the annual average versus the fraction of publications authored by female faculty in chemistry departments.} See the caption of Fig.~\ref{fig:collaboration:pub_coau_cheme} for details.}
\label{fig:collaboration:pub_coau_ch}
\end{figure}

\clearpage

\begin{figure}[t]
\includegraphics[width=\textwidth]{figures/collaboration_pub_coau_eco}
\caption[Average co-authors and publications in ecology.]{\textbf{Correlation between the average number of co-authors corrected for the annual average versus the fraction of publications authored by female faculty in ecology departments.} See the caption of Fig.~\ref{fig:collaboration:pub_coau_cheme} for details.}
\label{fig:collaboration:pub_coau_eco}
\end{figure}

\clearpage

\begin{figure}[t]
\includegraphics[width=\textwidth]{figures/collaboration_pub_coau_matsci}
\caption[Average co-authors and publications in material science.]{\textbf{Correlation between the average number of co-authors corrected for the annual average versus the fraction of publications authored by female faculty in materials science departments.} See the caption of Fig.~\ref{fig:collaboration:pub_coau_cheme} for details.}
\label{fig:collaboration:pub_coau_matsci}
\end{figure}

\clearpage

\begin{figure}[t]
\includegraphics[width=\textwidth]{figures/collaboration_pub_coau_psy}
\caption[Average co-authors and publications in psychology.]{\textbf{Correlation between the average number of co-authors corrected for the annual average versus the fraction of publications authored by female faculty in psychology departments.} See the caption of Fig.~\ref{fig:collaboration:pub_coau_cheme} for details.}
\label{fig:collaboration:pub_coau_psy}
\end{figure}

\clearpage

\begin{figure}[t]
    \begin{subfigure}{0.99\textwidth}
        \includegraphics[width=\textwidth]{figures/movies_active_elements}
        \phantomcaption
        \label{fig:movies:active_elements_a}
    \end{subfigure}
    \begin{subfigure}{0\textwidth}
        \phantomcaption
        \label{fig:movies:active_elements_b}
    \end{subfigure}
    \begin{subfigure}{0\textwidth}
        \phantomcaption
        \label{fig:movies:active_elements_c}
    \end{subfigure}
    \begin{subfigure}{0\textwidth}
        \phantomcaption
        \label{fig:movies:active_elements_d}
    \end{subfigure}
    \begin{subfigure}{0\textwidth}
        \phantomcaption
        \label{fig:movies:active_elements_e}
    \end{subfigure}
    \begin{subfigure}{0\textwidth}
        \phantomcaption
        \label{fig:movies:active_elements_f}
    \end{subfigure}
\caption[Evolution of female representation as directors and producers.]{\textbf{Evolution of female representation as directors and producers}. Number of (\textbf{a}) females and (\textbf{b}) males directing a movie for the first time (entry) or for the last time (exit) for U.S.-produced movies. For females, entries almost exactly balance exits. For males, between 1920 and 1940, the number of entries systematically exceeds the number of exits. (\textbf{c}) Percentage of females among active movie directors. The more equitable condition of the early 1900s (dashed line, approximately \activeDirEarlyMean{}\%) was only reached again in \activeDirYearRecover{}, having remained below half of that level for \activeDirYearsBelow{} years (dash-dotted line). Number of (\textbf{d}) females and (\textbf{e}) males producing a movie for the first time (entry) or for the last time (exit) for U.S.-produced movies. For females, entries almost exactly balance exits until 1975, at which point entries exceed exits. For males, between 1920 and 1940, the number of entries systematically exceeds the number of exits. (\textbf{f}) Percentage of females among active movie producers. The more equitable condition of the early 1900s (dashed line, approximately \activeProdEarlyMean{}\%) was only reached again in \activeProdYearRecover{}, having remained below half of that level for \activeProdYearsBelow{} years (dash-dotted line).}
\label{fig:movies:active_elements}
\end{figure}

\clearpage

\begin{figure}[t]
\includegraphics[width=\textwidth]{figures/movies_producers}
\caption[Historical trends of female representation in different producer role.]{\textbf{Historical trends of female representation in different producer roles}. Percentage of females among different active movie producer roles over time.}
\label{fig:movies:producers}
\end{figure}

\clearpage

\begin{figure}[t]
\includegraphics[width=\textwidth]{figures/movies_directors_pref}
\caption[Actor preferences according to director's gender.]{\textbf{Actor preferences according to director's gender}. Historical trend of the mean percentage of female actors cast in movies directed by a female (orange) or a male (purple). Female directors have a significantly higher preference for female actors than male directors (Mann-Whitney test, $U = \genderDirU{}$, $p \genderDirUPval{}$). Shaded regions represent the 99\% confidence bands calculated using the Clopper-Pearson \cite{Clopper1934} method under a binomial process for selecting a movie's cast (see Appendix \ref{sec:methods:movies} for details).}
\label{fig:movies:directors_pref}
\end{figure}

\clearpage

\begin{figure}[ht]
\includegraphics[width=\textwidth]{figures/lognormal_mu_publications}
\caption[Dependence of $\hat{\mu}$ on number of publications.]{\textbf{Dependence of $\hat{\mu}$ on number of publications at the individual level}. We fit the model to 1,000 randomized subsets of each researcher's publication list and compare the $\hat{\mu}$ obtained from fitting each subset of 10, 50, and 100 publications with the $\hat{\mu}$ associated with the complete publication list. Then, for each researcher and subset size, we calculate a z-score using the mean and standard deviation of the ``sub-$\hat{\mu}$''. For $N_p \geq$ 50, the dependence on sample size is negligible for most researchers. Researchers with $N_p <$ 100 are omitted from the calculation on the subset of size 100.}
\label{fig:lognormal:mu_publications}
\end{figure}

\clearpage

\begin{figure}[t]
\includegraphics[width=\textwidth]{figures/lognormal_sigma_publications}
\caption[Dependence of $\hat{\sigma}$ estimates on number of publications.]{\textbf{Dependence of $\hat{\sigma}$ estimates on number of publications at the individual level}. We use the same procedure as in Fig.~\ref{fig:lognormal:mu_publications}, except here we show the results for the dependence of $\hat{\sigma}$ on sample size. Estimates of $\hat{\sigma}$ are more dependent of sample size than $\hat{\mu}$. However, as in the case of $\hat{\mu}$, the dependence of $\hat{\sigma}$ on sample size decays rapidly with increasing sample size. Researchers with $N_p <$ 100 are omitted from the calculation on the subset of size 100.}
\label{fig:lognormal:sigma_publications}
\end{figure}

\clearpage

\begin{figure}[t]
\includegraphics[width=\textwidth]{figures/lognormal_manipulation_cites}
\caption[Susceptibility of impact measures to manipulation.]{\textbf{Susceptibility of impact measures to manipulation}. We used the same procedure as in Fig.~\ref{fig:lognormal:manipulation_papers}, except here we show the required number of publications with self-citations that researchers need to publish in order to increase their indicators. Other details are the same as in Fig.~\ref{fig:lognormal:manipulation_papers}.}
\label{fig:lognormal:manipulation_cites}
\end{figure}

\clearpage

\begin{figure}[t]
\includegraphics[scale=0.4]{figures/lognormal_sigma_f_comparison}
\caption[Comparison of $\hat{\sigma}$ and $f_s$ across departments, journals, and researchers.]{\textbf{Comparison of $\hat{\sigma}$ and $f_s$ across departments, journals, and researchers}. We show the maximum likelihood fitted $\hat{\sigma}$ (\textbf{top}) and the fraction of secondary publications (\textbf{bottom}) for chemistry departments and chemistry journals in select years, and for all chemistry researchers in our database. The black horizontal dashed lines mark the value of the corresponding parameter for the \textit{Journal of the American Chemical Society} in 1995. For clarity, we do not show $\hat{\sigma}$ for 19 journals and 9 researchers that are outliers.}
\label{fig:lognormal:sigma_f_comparison}
\end{figure}
