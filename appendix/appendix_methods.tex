\chapter{Methods}
\label{ch:methods}

\section{\nameref*{ch:collaboration}}
\label{sec:methods:collaboration}

\subsection{Co-author names matching}

To calculate the number of distinct co-authors for a researcher, we used the following procedure. For each researcher, we maintain a set of standardized co-author names. For each co-author name, we convert the name to a string of last name and first name initials. For example, a co-author named ``Jane Linda Smith" will be converted to ``Smith JL". For each publication, we standardize the names of the co-authors, and add them to the set. We finally count the number of elements in the set.

Note that using this procedure, we treat ``Jane Linda Smith" and ``Jane Lily Smith" as the same name, because they are both converted to the string ``Smith JL". Also, we treat ``Jane Linda Smith" and ``Jane Smith" as different names, since the former is converted to ``Smith JL", while the latter is converted to ``Smith J". In reality, for a single author's co-authors, the probability for either case to happen is very small, hence the error rate of our procedure is very low.


\subsection{Confidence interval for the survival curve of total number of distinct co-authors}

We use matched sampling to obtain the confidence interval for the survival curve of total number of distinct co-authors. We consider the null hypothesis that there is no difference in the total number of co-authors between females and the males with similar number of publications. To construct the confidence interval, we generate samples of $N_F$ males, where $N_F$ the number of females in our dataset. For a female with $n_F$ publications, we select a male whose number of publications falls in the range of $[0.8\;n_F,~1.2\;n_F]$, a range small enough to produce good matches but large enough that there is at least one match. We then compute the survival curve for the obtained sample of male authors. We obtain the confidence interval by repeating this procedure 1,000 times.

The procedure is similar for the null hypothesis that there is no difference in the total number of co-authors between females and the males with equal number of publications, except that the sample of males consists of males who have the same number of publications as the females.


\subsection{Measuring gender difference in the distribution of collaboration opportunities}

We use two methods, the Gini coefficient and the disparity index, to measure how homogeneously each author distributes all her/his collaboration opportunities among her/his co-authors. A high Gini coefficient or disparity index indicates inhomogeneity of collaboration frequency distribution, where the author collaborates highly frequently with only a small portion of her/his co-authors, but only a few times with each of the remaining majority. Thus, this author has a high propensity to concentrate her/his collaboration opportunities on a few co-authors. A low Gini coefficient or disparity index indicates that the author collaborates with each of her/his co-authors about equally frequently.

\textit{Gini coefficient}. Consider author $a$ with $n_c$ co-authors. For each co-author $c_i$ of $a$, we count the times of collaboration between $a$ and $c_i$, $y_i$. That is, the number of publications $a$ has co-authored with $c_i$. We next arrange $y_i$ in non-decreasing order, where $y_i \leq y_{i+1}$. The Gini coefficient of author $a$ is calculated as
\begin{equation}
G(a) = \frac{2\sum \limits_{i=1}^{n_c} iy_i}{n_c \sum \limits_{i=1}^{n_c} y_i} - \frac{n_c+1}{n_c}\;\;.
\end{equation}

\textit{Disparity index}. We first calculate the weight of collaboration (link) between $a$ and $c_i$ as given by Newman \cite{Newman2004},
\begin{equation}
w_{ac_i} = \sum \limits_{j=1}^{k_{c_i}} \frac{1}{l_j-1}\;\;,
\end{equation}

\noindent
where $k_{c_i}$ is the number of publications authored by $a$ and $c_i$ together, and $l_j$ is the number of co-authors in publication $j$. Then we calculate for $a$ the summation of the weights of collaboration (strength),
\begin{equation}
s_a = \sum \limits_{i=1}^{n_c} w_{ac_i}\;\;.
\end{equation}

\noindent
Finally, the disparity index is calculated as
\begin{equation}
\Upsilon(a) = \sum \limits_{i=1}^{n_c} \left( \frac{w_{ac_i}}{s_{a}} \right)^2 n_c\;\;.
\end{equation}

\noindent
We obtain the sample of Gini coefficients for female authors, $\{G_F\}$, and that for male authors, $\{G_M\}$. We then can obtain the significance of the difference between the two samples, by performing a Kolmogorov-Smirnov test on the cumulative distribution function curves of the two samples. We perform the same hypothesis test for $\{\Upsilon_F\}$ and $\{\Upsilon_M\}$.

