\graphicspath{ {./figures/ratio/} }

%%%%%%%%%%%%%%%%%%%%%%%%%%%%%%%%%%%

The experiments detailed in this section were conceived, designed, and conducted by my collaborators, most notably Nicol'{a}s Pel'{a}ez and Jean-Francois Boisclair Lachance. They are included here for completeness.

\section{Genetics}
\label{appendix:ratio:genetics}

The recombineered \textit{pnt-gfp} BAC transgene inserted into the VK00037 landing site was previously described in Boisclair-Lachance et al. (2014). Alleles $pnt^{\Delta 88}$ \cite{ONeill1994a} and $pnt^2$ (Bloomington Stock 2222) were used to render the endogenous \textit{pnt} gene null in the presence of \textit{pnt-gfp}. A single copy of \textit{pnt-gfp} rescued $pnt^{\Delta 88}/pnt^2$ to full viability and fertility (Fig. \ref{fig:ratio:figS1}A). Cell nuclei of developing eye-antennal discs were marked by recombining \textit{H2Av-mRFP} (Bloomington stock 23651) with $pnt^2$. Experiments measuring wild type dynamics of \textit{Pnt-GFP} were done by dissecting eye discs from white prepupae carrying $w^{1118}$ ; $pnt\hyphy gfp / pnt\hyphy gfp$ ; $pnt^{\Delta 88}/pnt^2$, $H2Av\hyphy mRFP$. \textit{Pnt} isoform-specific expression was detected using enhancer traps \textit{HS20} (gift from B. Shilo) and $pnt^{1277}$ (Bloomington stock 837), which report \textit{PntP1} and \textit{PntP2} transcription respectively by expressing LacZ \cite{Scholz1993}. \textit{pnt-gfp} and \textit{pnt} isoform-specific expression were compared in white prepupae carrying $w^{1118}$; $pnt\hyphy gfp/pnt\hyphy gfp$; $HS20/+$ and $w^{1118}$;$pnt\hyphy gfp/pnt\hyphy gfp$; $pnt^{1277}/pnt^{1277}$. \textit{Pnt} gene dosage experiments were done using $w^{1118}$; $pnt\hyphy gfp/+$ ; $pnt^{\Delta 88}/pnt^2$ (1x pnt) and $w^{1118}$; $pnt\hyphy gfp/pnt\hyphy gfp$ ; $pnt^{\Delta 88}/pnt^2$ (2x pnt). Notch activity was conditionally reduced using the $N^{ts1}$ temperature sensitive allele \cite{Shellenbarger1975}. $N^{ts1}/N^{ts1}$ ; $pnt\hyphy gfp/+$ animals were raised at the permissive temperature (18 \textdegree{}C) and shifted to the restrictive temperature (28.5 \textdegree{}C) as third instar larvae for 24 h. Animals exposed to the restrictive temperature that were transferred back to the permissive temperature had roughened eye phenotypes and a notched wing phenotype as adults, consistent with effective inhibition of Notch activity. Control larvae of the same genotype were grown at the permissive temperature until dissection. Both control and heat-treated larvae were sexed and only \textit{N} hemizygote males carrying $N^{ts1}/Y$ ; $pnt\hyphy gfp/+$ were dissected as white prepupae. EGFR activity was conditionally reduced by placing the null allele $egfr^{f24}$ - also known as $egfr^{CO}$ \cite{Clifford1989} \textit{in trans} to the thermo-sensitive allele $egfr^{tsla}$ \cite{Kumar1998}, as previously described \cite{Pelaez2015a}. The genotype was $w^{1118}$; $egfr^{tsla}$, $pnt\hyphy gfp/egfr^{24}$, $pnt\hyphy gfp$. Ras activation was achieved using a transgene expressing a $Ras1^{V12}$ mutant and driven by a 3x\textit{sev} enhancer and promoter \cite{Fortini1992} as previously described \cite{Pelaez2015a}. \textit{Pnt-gfp} in the Ras mutant background was measured using $w^{1118}$; $pnt\hyphy gfp$, $Sev>Ras^{v12}/pnt\hyphy gfp$;$pnt^2$,$H2Av\hyphy mRFP/+$. Controls animals carried $w^{1118}$; $pnt\hyphy gfp/pnt\hyphy gfp$, $pnt^2$, $H2Av\hyphy mRFP/+$. \textit{Yan} mutant eye clones were generated using the $yan^{833}$ null allele \cite{Webber2013}, \textit{ey$>$FLP} and the FRT40 crossover point. $Pnt^+$ tissue was labeled using the clonal marker $Ubi>mRFP_{nls}$ (Bloomington Stock 34500). Developing eyes were dissected from white prepupae carrying $w$, $ey>FLP$; $pnt\hyphy gfp$, $yan^{833}$, $FRT40A/pnt\hyphy gfp$, $Ubi>mRFP_{nls}$, $FRT40A$. Control discs to measure the GFP-mRFP fluorophore bleed-through were obtained from flies carrying $w$, $ey>FLP$; $pnt\hyphy gfp$, $Ubi>mRFP_{nls}$, $FRT40A/pnt\hyphy gfp$, $FRT40A$ or $w$, $ey>FLP$; $pnt\hyphy gfp$, $Ubi>mRFP_{nls}$, $FRT40A/CyO$.

\section{Immunohistochemistry}
\label{appendix:methods:ratio:immunohistochemistry}

Unless otherwise noted, Pnt-GFP and Yan were measured in developing animals raised at 21 \textdegree{} C, selected as white prepupae, and subsequently aged in humid chambers for 5-10 h. Eye-antennal discs were dissected in PBS, and fixed in 4 \% (w/v) paraformaldehyde/PBS for $\sim$45 min. Endogenous Yan protein was detected with the mouse monoclonal anti-Yan antibody 8B12 (Developmental Studies Hybridoma Bank, 1:200 dilution) and the secondary goat anti-mouse Pacific Blue antibody (Life Technologies, 1:200 dilution). Expression from the \textit{HS20} and $pnt^{1277}$ enhancer traps was detected using mouse anti-$\beta$-galactosidase 40-1a (Developmental Studies Hybridoma Bank, 1:50 dilution). H2Av-mRFP was used as a nuclear marker as previously described in Pel\'{a}ez et al. (2015). Discs were incubated in 1:1 (v/v) PBS:VectaShield (Vector Laboratories) for 45 min, then in 100\% VectaShield for an additional 45 min before mounting.

For experiments using \textit{yan} mutant clones, $N^{ts}$, or $EGFR^{ts}$ alleles, nuclei were stained with a 4',6-diamidino-2-phenylindole (DAPI) nuclear marker. Samples were fixed in 4\% paraformaldehyde, rinsed with PBS-Tween 0.5\%, and permeabilized with PBS-Triton X-100 0.1\% for 20 minutes at room temperature. Permeabilization was important to allow DAPI penetration without perturbing the fluorescence of the Pnt-GFP protein. After permeabilization, eye discs were incubated in a blocking solution containing PBS-Tween 0.1\% and 1\% normal goat serum for 30 minutes at room temperature. Primary and secondary antibodies were incubated each for 2 hours at room temperature. Antibodies used with DAPI were: mouse anti-Yan 8B12 (DHSB, 1/500) and goat anti-mouse Cy3 (1/2000, Jackson Immunoresearch). Discs were mounted in 0.5\% n-propyl-gallate, 0.1M Tris pH 8.0 and 90\% glycerol.

Samples were kept in the dark at -20 \textdegree{}C and imaged no later than 18-24 hr after fixation. In all cases, 1024 x 1024 16-bit images were captured using either a Zeiss LSM880 or a Leica SP5 confocal microscope equipped with 40X oil objectives. During imaging, discs were oriented with the equator parallel to the x-axis of the image. Optical slices were set at 0.8µm slices (45-60 optical slices) with an additional digital zoom of 1.2-1.4 to completely image eye discs from basal to apical surfaces. Images recorded a region of at least 6 rows of ommatidia on each side or the dorsal-ventral eye disc equator. All discs for a given condition were fixed, mounted, and imaged in parallel to reduce measurement error.
