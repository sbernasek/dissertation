The research presented throughout this dissertation spawned several computational tools that might catalyze future research. All of these resources have been made freely available online, with the hope that their continued use and development will benefit the broader community of quantitative biologists. The list below describes each tool and its high level functions. With the exception of FlyEye Silhouette, all software is accessible via GitHub repositories mirrored between both \href{https://github.com/sebastianbernasek/}{my personal account} and the \href{https://github.com/amarallab}{Amaral} and \href{https://github.com/bagherilab}{Bagheri} lab accounts. These repositories contain high level API documentation in addition to a series of Jupyter notebooks that walk the user through a series of usage examples. 
% [noitemsep,topsep=0pt,parsep=0pt,partopsep=0pt]

\begin{itemize}[leftmargin=*,topsep=10pt, itemsep=10pt]
  
  % FLYEYE SILHOUETTE
  \item \textbf{FlyEye Silhouette}: \url{http://silhouette.amaral.northwestern.edu}
  \newline
  GUI-based MacOS application for segmentation, quantification, and annotation of cell nuclei in the \textit{Drosophila} eye imaginal disc. Developed by Helio Tejedor in the Amaral lab.
  
  % FLYEYE ANALYSIS
  \item \textbf{FlyEye Analysis}: \url{https://github.com/sebastianbernasek/flyeye}
  \newline
  Python framework for analyzing data generated using FlyEye Silhouette. Core features include inference of cell developmental ages and analysis of the resultant expression dynamics. Also provides tools to quantify expression heterogeneity and spatial patterns.
  
  % FLYEYE ANALYSIS
  \item \textbf{FlyEye Clones}: \url{https://github.com/sebastianbernasek/clones}  
  \newline
  Python framework for automated mosaic analysis of \textit{Drosophila} eye imaginal discs. Core features will be integrated with future versions of FlyEye Silhouette. 
  
  % FLYEYE SYCLONES
  \item \textbf{FlyEye SyClones}: \url{https://github.com/sebastianbernasek/syclones}
  \newline 
  Python framework for generating synthetic microscopy data that mimic key features of mosaic eye imaginal discs.
  
  % PolyTF BINDING
  \item \textbf{PolyTF Binding}: \url{https://github.com/sebastianbernasek/binding}
  \newline 
  Python framework for simulating the equilibrium occupancy of DNA binding sites by one or more polymerizing transcription factors. Utilizes a C backend that efficiently enumerates all possible microstates in a recursive fashion, enabling nested parallelization of the main computational bottleneck. For systems of two or more transcription factors, the implementation confers a major performance advantage over the sequential enumeration strategy proposed by the manuscript that inspired the model \cite{Hope2017}.
  
  % GENESSA
  \item \textbf{GeneSSA}: \url{https://github.com/sebastianbernasek/genessa}
  \newline
  Python framework for exact stochastic simulation of gene regulatory network dynamics \cite{Gillespie1977}. Simulations are executed by a C backend optimized for performance on networks with a narrow scope of pre-defined reaction propensity functions. The limited scope is by design; GeneSSA prioritizes computational efficiency at the expense of flexibility by explicitly hard coding a set of functional forms. This design places GeneSSA among the most performant implementations of the exact stochastic simulation algorithm for several common types of GRNs. The framework may be (and has been) extended to include additional kinetic formulations as they are required.
  
\end{itemize}
