\chapter{Conclusion}
\label{ch:conclusion}

In this dissertation, I have presented a rigorous analysis of gender disparities in creative teams. I first analyzed the differences in collaboration patterns between male and female STEM researchers. Then, I studied the origins of gender discrimination in the U.S. movie industry. I have also presented a framework to quantify scientific impact of individual researchers and academic institutions.

My work is noteworthy in that all results are derived from rigorous statistical analysis of large-scale datasets. I used the Web of Science database of scientific publications when quantifying the effect of gender diversity in scientific collaborations and the impact of scientific publications; and the Internet Movie Database when quantifying the effect of gender diversity in movie productions.

In \autoref{ch:collaboration}, I studied gender differences in scientific collaborations. I first proved that, even though female researchers have less distinct collaborators, this is only due to the fact that females publish less than males and have shorter career lengths. I then showed that, despite these disadvantages, females actually have a higher propensity to engage in novel collaborations, suggesting their work to be of higher impact than that of males. Finally, I presented evidence of female exclusion from genomics, a sub-disciplines of molecular biology.

In \autoref{ch:movies}, I present evidence for how females have been discriminated against in the U.S. movie industry. Namely, I demonstrated that during the years of the Hollywood studio system, female representation among actors, directors, and producers dropped by more than half. This under-representation may be at least partially responsible for today's observed gender imbalance in the movie industry as I also found that the gender diversity of a movie's producers influences both the gender of the director and the gender composition of the cast, and that female directors have a statistically significant preference for more gender-balanced casts. Additionally, I showed that female directors are over-represented in \overGenres{}, and under-represented in seven other genres \underGenres{}. Finally, I found that higher than average female representation became concentrated in higher budget movies during the studio system, but in the 1960s higher than average female representation shifted to the movies with the lowest budgets.

In \autoref{ch:lognormal}, I put forth the notion that scientific publications have a latent ``citability'' that can be estimated using the asymptotic number of citations. Specifically, I determined that the asymptotic number of citations $n_a$ to sets of publications by a researcher or associated with an academic department can be described by a discrete lognormal distribution. I performed a principled statistical analysis of the properties of this distribution and showed that the mean citability, $\mu$, can be used as an unbiased bibliometric indicator of scientific impact for researchers, departments, and journals. Furthermore, $\mu$ is resistant to manipulation, unlike other popular indicators such as the \emph{h}-index, and can be well approximated by the median of the logarithm of $n_a$.


\section{Societal implications of gender biases in creative teams}

Collaborations decrease many barriers towards producing works of high impact which benefits all creators involved in the process. I have determined that gender has a profound effect in creative teams. While other researchers have also reported gender effects in teams, my research has several distinctive features. First, I leveraged the power of ``Big Data'' in order to avoid many sampling biases of small datasets that can lead to inaccurate conclusions. Second, I controlled for some inherent complexities in my systems of study --- scientific collaborations and movie productions --- that can make it difficult to draw the correct inferences, regardless of dataset size. For instance, a direct analysis of co-authorship patterns in collaborations would lead to the erroneous conclusion that male researchers collaborate more than females; and only by correcting for differences in publication volume and shorter career lengths can we uncover the true relation (Figs.~\ref{fig:collaboration:coau}, \ref{fig:collaboration:coau_pubs}). Similarly, only by accounting for the fact that there are very few female movie directors can we show the rich genre differences across movie directors (Fig.~\ref{fig:movies:genre_c}). My findings illustrate the need to always consider the context of where the data collected when performing any analysis.

While my study on scientific collaborations is limited to U.S. faculty members in seven distinct STEM disciplines, it could easily be extended to all scientific disciplines where collaborations are the norm. Thus we could precisely determine how field-dependent the gender effects in collaborations are. Large gender differences across sub-disciplines of the same large discipline could indicate the presence of strong gender discrimination, such as the case of molecular biology (Fig.~\ref{fig:collaboration:pubs_molbio}). Conversely, academic practices in disciplines showing very minimal gender differences in collaborations warrant a deeper look as they may be promoting female participation in science.

Many researchers investigate the factors contributing to the the gradual loss of female representation along the academic career path --- the ``leaky pipeline''. The findings from studies can be combined with the present work to create guidelines or policies that ensure proper institutional support for both genders. For instance, highlighting the academic achievements of female researchers and creating inclusive environments for female postdoctoral students and faculty members could foster an increase in female representation in science. Indeed, evidence indicates that increasing the visibility of female leaders in careers of low female representation, such as business and politics, has a positive contribution to female advancement and can decrease gender biases \cite{Dean2008,Jalalzai2013,Zaichkowsky2014}.

Conversely, my study of gender representation in the U.S. movie industry clearly shows how the lack of female role models can have a strong negative impact on gender diversity: while the early U.S. movie industry had a relatively high gender diversity --- females composed about one third of the cast in the typical at the time --- once the studio system was established, many females either left the industry or were forced to leave, especially those working behind the camera (Fig.~\ref{fig:movies:comparison_a}). My results suggest that the accumulation of power in the hands of the few white male leaders of the big Hollywood studios lead to females being excluded from the industry. Furthermore, this negative influence can carry a strong inertia, as evidenced by the fact that, after the studio system was dissolved, it took decades for female representation to recover to pre-studio levels.

If there is power consolidation in sectors experiencing big growth, we should create incentives for teams at the top to remain as diverse as possible so as to avoid instituting biases --- not just gender-related --- that can take years to dispel. The discipline of computer science provides another illustrative example of this phenomenon. Programming pioneers such as Ada and Grace Hopper certainly made the nascent field appealing to females \cite{Sydell2014}. Then, with the creation of the PC, there was the opportunity to make the field accessible to the general public. Unfortunately, the product was almost exclusively marketed towards young males, which lead to a surge of male interest in the field \cite{Margolis2003}. Despite educational reforms and focus groups aimed at increasing female interest in the discipline, the stereotypical computer programmer is still overwhelmingly a young male.

Finally, it is worth noting that the factors I identified as possible causes of gender biases may also explain under-representation of other minorities in creative teams. For example, given the appropriate datasets, my analyses can be adapted to study the effect of racial or ethnic diversity in movie success, or to understand how cultural diversity of ideas affect the impact of a scientific publication.


\section{Guidelines to quantify the impact of creative works}

The goal of science is to accurately quantify and measure natural phenomena. For this reason alone we should move away from using heuristics and \textit{ad-hoc} measures if we want to measure the impact of science itself in a rigorous way. Upon closer look, the disadvantages of bibliometric indicators such us the \emph{h}-index and Journal Impact Factor far outweigh their touted ease-of-use and simple interpretation. For instance, the fact that the \emph{h}-index increases monotonically over time makes it unsuitable to compare researchers at different career stages, as it penalizes younger researchers with fewer publications. Moreover, because of its dependence on publication volume, the \emph{h}-index can be boosted if researchers spread their results over many publications, in effect encouraging quantity over quality of scientific research. This incentive system has consequences for hiring committees and funding agencies that may use bibliometric indicators as a first screen of their potentially hundreds of applicants. The solution to this problem is not to propose corrections to the \emph{h}-index or more-complex \textit{ad-hoc} indicators but instead use a principled, data-driven approach to quantify scientific impact.

In contrast to most existing heuristic bibliometric indicators, my proposed framework is grounded on the latent citability of a publication and as such can be used to systematically characterize the impact of any set of publications, be it those authored by a researcher, associated with a department, or published in a journal. Being able to place work by all these three entities in the same scale --- the expected citability --- is especially relevant when there is uncertainty about a researcher's impact. It may be difficult to directly assess the impact of a young researcher with few publications or an unknown researcher from a different discipline, but the expected citability of the journals where they have published work or their institutional address will provide a good indication of expected researcher impact. Conversely, work by researchers with high expected citability may be noteworthy even if it is published in unknown journals. Thus, my framework will enable hiring committees and funding agencies to speed up their evaluating process while simultaneously be confident that they are making sound decisions.

To create a general framework of impact of creative works, it is not enough to have a solid mathematical foundation. Any proposed model must be carefully validated against a representative dataset. For the case of scientific impact, I validated the citability framework against hundreds of thousands of publications across different scientific disciplines. With large datasets from various domains becoming more accessible than ever before, my approach can be applied to not only quantify scientific impact in other disciplines but the impact of most creative works in areas outside of science.

One such domain is the movie industry. In the U.S., movies that are deemed ``culturally, historically, or aesthetically significant'' to the country are preserved in the National Film Registry (NFR) \cite{Congress2017}. For U.S. productions, induction into the NFR is perhaps the closest indicator of latent movie impact. Indeed, Wasserman et al.\ recently showed that the \textit{long-gap citations}, i.e., the number of times that a movie is referenced in other movies that are 25+ years younger is a good predictor of induction into the NFR \cite{Wasserman2015}. Thus these researchers demonstrated that \textit{long-gap citations} constitute a quantitative, principled indicator of movie significance, much like the expected citability for publication impact.
