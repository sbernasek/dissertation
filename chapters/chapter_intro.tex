\chapter{Introduction}
\label{ch:intro}

Humans are a cooperative species. Governments must engage in trade deals to provide their respective citizens with more choices of goods and exotic delicacies. International and private institutions regularly fund scientific collaborations involving researchers from different universities and countries \cite{Bordons1996}. Companies strategically share resources and partners to create surprising new innovations \cite{Ahuja2000,Dyer2002}. Artists who work in close proximity act share ideas and act as each other's critic and fan, thereby improving the overall quality of their works \cite{Uzzi2005}. As illustrated by these examples, we band together to find solutions to problems that no single individual could solve alone. Indeed, a ``collective intelligence'' can emerge for a group of individuals that is not simply the sum of individual intelligence \cite{Woolley2010}.

Despite the advantages of working as a team \cite{Gajda2004,Guimera2005,Uzzi2005,Wuchty2007,Katzenback2008}, personal and hierarchical differences between individuals and institutions can create conflicts, insecurity, and miscommunication that detract from the team's effectiveness \cite{Edmonson1999,Jehn1999,Cummings2005,Jones2008}. The exact effect of individual characteristics such as age, gender, or location on team impact is an area of active research.

Gender is a particularly relevant characteristic. Even though the general population is nearly gender-balanced, this is not observed in most sectors of society. While some argue that some professions are more suited to a single gender, the fact is that gender diversity is increasingly regarded as a desired condition by many institutions and corporations. Indeed, a prolonged gender imbalance in a given occupation can turn to unconscious bias that, over time, will give rise to the unhealthy stereotype that only males (or females) are suited for that job \cite{Editorials2013}.

In science, technology, engineering, and mathematical (STEM) disciplines, researchers have shown that females make for better collaborators than males \cite{Berdahl2005,Kummerli2007,Bart2013} and that mixed-gender teams produce higher impact works than single-gender groups \cite{Campbell2013}, others have reported that males publish more \cite{Kyvik1996} and are more prolific collaborators than females \cite{Lee2005,Abramo2013}. Given the current gender gap in STEM disciplines it is vital that we understand how researchers' gender facilitates or hinders the effectiveness of collaborations.

However, studying gender discrimination in science teams is complicated by claims that there are innate mathematical and logical ability differences between genders \cite{Voyer1995,Hyde2008}, and that females choose to leave academia to raise children \cite{Ceci2011} or to pursue a scientific career in other industries \cite{Etzkowitz2011}. One profession where no such arguments can be made is acting. Indeed, the movie industry gives the same accolades to female and male actors. Moreover, the fact that movie productions are usually just a few months long and that actors can go several years without appearing in a motion picture make acting more amenable to childcare than most other careers.

Yet, there is evidence for gender discrimination in the movie industry. While females are present in nearly all movies, action movies are typically associated with males, whereas romance movies are more closely identified with females \cite{Smith2014,Wuhr2017}. Furthermore, Hollywood has an insidious gender wage gap, as recently brought to light by some of the industry's most famous actors and actresses \cite{Adamczyk2016,Gonzales2016,Sollosi2017}. The origin of this gender discrimination and the effect of gender diversity in movie-making teams are still unsettled questions.

To fully determine the success of creative teams it is not enough to study their gender diversity. We must also analyze how the work produced by teams is perceived by their peers. Scientific collaborations create publications whose impact can be quantified using bibliometric indicators. Unfortunately, despite the rather large number of \textit{ad-hoc} bibliometric indicators of scientific impact proposed in recent years \cite{Hirsch2005,Egghe2006b,Jin2007a,Alonso2009a,Franceschini2010,Bharathi2013,Lando2014,Xu2015,Frittelli2016,Hutchins2016}, there have been surprisingly few attempts to develop a rigorous framework that reliably quantifies scientific impact \cite{Radicchi2008,Stringer2008,Radicchi2009,Petersen2011,Wang2013a}. Such a framework can be used to promote science of excellent quality, with the capability to promote innovation, economic growth, and social well-being.

In this dissertation, I present a quantitative, large-scale study of the effect of gender diversity in creative teams, coupled with a rigorous framework to quantify the impact of scientific works.


\section{Gender disparities in scientific collaborations}

Collaborations bring many benefits to all scientists involved. Studies show that collaborations can decrease experimental costs \cite{Dyer2002}, increase researcher productivity \cite{Wood1991,Bordons1996,Gajda2004} and creativity \cite{Ahuja2000,Uzzi2005}. Moreover, teams have a greater chance of producing publications with higher impact than individuals \cite{Wuchty2007}, especially if they constitute novel collaborations \cite{Guimera2005,Katzenback2008}.

Given that collaborations can deeply impact researchers' careers, it is vital to understand the individual factors that enable a collaboration to be successful such as researcher nationality \cite{Schubert2006}, institute, \cite{Cummings2005}, discipline \cite{Jones2008}, or gender \cite{West2013}. The effect of this last factor is of particular interest. On one hand, some studies revealed that, compared to males, females have fewer single-author publications than males \cite{Kyvik1996}, prefer to work in less hierarchical structures \cite{Berdahl2005}, show less self-interest \cite{Bart2013}, and are more cooperative \cite{Kummerli2007}, suggesting females make for better collaborators. On the other hand, other researchers showed that males can be more productive than females \cite{Wanner1981,Lee2005} and have more international collaborations \cite{Abramo2013}. However some of these apparently contradictory results rely on small samples or self-reported surveys and thus have small statistical power.

Furthermore, female researchers are at a disadvantage in nearly all science, technology, engineering, and math (STEM) disciplines. Females comprise only a small percentage of faculty members \cite{Duch2012} in STEM and there is a growing gender gap with advancing levels of science specialization \cite{Leadley2009}, the so-called ``leaky pipeline'' phenomenon. Several researchers also report that female faculty suffer systemic and selective pressures creating a ``glass ceiling'' that prevents career advancement \cite{Menges1983,Jacobs2005,Carnes2008,Wolfinger2008,Moss-Racusin2012} and that females are more risk-averse than males \cite{Harris2006}.

A proper study of gender diversity in scientific collaborations should take structural factors such as academic positions and publication volume into consideration. Indeed, after controlling for age, discipline, and career stage, Bozeman et al.\ find that females overall collaborate more than males after \cite{Bozeman2004,Bozeman2011}. Moreover, McDowell et al.\ find evidence for gender homophily in collaborations among economists \cite{McDowell1992}, i.e., researchers prefer to collaborate with others of the same gender. Thus, the presence of gender homophily suggests that females have fewer opportunities for collaboration \cite{Kegen2013}, which could help explain some of the apparently contradictory results on gender differences in collaborations. A systematic, large-scale study clarifying the role that gender diversity plays in scientific collaborations would go a long way towards understanding productivity differences between male and female researchers in STEM disciplines.


\section{Gender discrimination in movie productions}

Movies have the power to make us afraid, laugh, cry, think, and even angry. Some actors can obtain a high level of notoriety from their movies which enable them to get cult-like followings \cite{Egan2013}, dictate fashion trends \cite{Berry2002}, and even exert political influence \cite{Nownes2012}. On the whole, the movie industry has an enormous impact on the world economy. In 2015, 708 movies were released worldwide, which generated US\$38 billion in revenue \cite{MPAA2015} and involved more than 600,000 direct jobs \cite{MPAA2017}.

A movie can be viewed as a collaborative act between several actors, producers, directors, screenwriters, and other crew members. Therefore it is reasonable to assume that individual characteristics have an effect on the impact of movie-making teams, perhaps even more so than individual characteristics on scientific collaborations, since even so-called ``one-actor'' movies often require tens of supporting crew as well as a director and one or several producers.

In principle, gender should not play a role in the effectiveness of movie production teams: outstanding female and male actors are both similarly laudable, and the fact that, on average, actors participate in a single movie production per year precludes the need for female actors to go on maternity leave. Yet, examples of female discrimination were abundant throughout much of the $20^{th}$ century \cite{Smith2014,Smith2017}. Females actors suffer from age \cite{Bazzini1997,Lincoln2004} and salary \cite{DePater2014} discrimination, and get less acting opportunities than their male counterparts \cite{Dean2008,Lutter2013}.

Several researchers have suggested that the emergence of the Hollywood ``studio system'' may have been at least partly responsible for the observed gender discrimination in the movie industry \cite{Lincoln2004,Smith-Doerr2010,Narayan2016,DePater2014}. In 1920, the five biggest studios in Hollywood (MGM, Paramount, Warner Bros., RKO, and Fox) banded together into a cartel that controlled every aspect of a motion picture, from casting of actors and hiring of directors and writers, all the way to distribution and exhibition of the final movie \cite{Deutelbaum1989}. The few leaders of the production companies composing the studio system --- white males such as Louis B. Mayer, David Sarnoff, David O. Selznick, or Jack Warner --- essentially gained absolute control over the Hollywood movie industry.

The studio system started to crumble when, in 1944, actress Olivia de Havilland successfully sued Warner Bros. to end long-term contracts in Hollywood \cite{DeHaviland1944}. This decision gave actors greater creative freedom to chose their projects. The studio system was finally disbanded in 1948, after the U.S. Supreme court it to be in violation of anti-trust laws \cite{US1948}.

Adverse effects of the studio system's policies continued to be felt for years after its dissolution. Female screenwriters were present at the start of the movie industry, and even though they were the minority gender, the average female screenwriter had the same visibility as the average male screenwriter \cite{Smith-Doerr2010}. However, with the establishment of the studio system, female screenwriters were quickly pushed to the background. Only recently have female TV and movie screenwriters started to gain recognition again \cite{Bielby1996,Bielby2009}.

Surprisingly, most studies performed so far on gender discrimination against actors, producers, directors, or screenwriters are either mostly qualitative, or consider only recently-released or highest-grossing movies. A comprehensive, large-scale analysis of historical patterns of female representation in the movie industry is still lacking. Such an analysis can yield valuable insights regarding the effect of gender diversity in movie productions.


\section{Scientific impact of published research}

The exponential growth of scientific literature in the past half century has all but strained researchers ability to keep up with recent developments. To choose what to browse, read or cite is now a very challenging task for researchers. Simultaneously, the scientific workforce also experienced a tremendous growth. In order to continue generating ever more specialized, high quality knowledge, universities, funding agencies, and reviewers need to be able to evaluate the creativity and productivity of researchers. Neither researchers nor evaluating entities have in-depth experts on all fields, therefore they need to rely on proxies or indicators of publication quality, researcher impact, etc.

Bibliometric indicators are measures that consider one or more of counts of scientific publications and citations received by them in the scientific literature, co-authorship and concentration within specific journals, journal prestige just to name a few \cite{Narin1996,Borgman2002,Vinkler2004}. The number of citations, in particular, represent a measure of the impact or influence of not only specific publications but scientific journals \cite{Garfield1972}, individual researchers \cite{Hirsch2005, West2010}, research groups \cite{VanRaan2006a}, institutions \cite{Molinari2008,Bornmann2012,Abramo2013} or even whole cities and nations \cite{King2004,Mazloumian2013,Zhang2013}.

Various bibliometric indicators have been proposed such as the notorious Journal Impact Factor \cite{Garfield1963} and the \textit{h}-index \cite{Hirsch2005} which measure the impact of scientific journals and individual researchers, respectfully. Yet, despite their growing numbers \cite{Egghe2006b,Jin2007a,Alonso2009a,Franceschini2010,Bharathi2013,Lando2014,Xu2015,Frittelli2016,Hutchins2016}, for the most part, existing bibliometric indicators constitute simple heuristics of citation counts and thus can be biased by career stage or publication volume, or be susceptible to manipulation \cite{MacRoberts1989, Narin1996, Cole2000, Glanzel2002, Borgman2002, Vinkler2004, Bornmann2007, Bornmann2008, Alonso2009, Castellano2009, Wilhite2012}. To address these issues, many researchers sought to develop bibliometric indicators that are unbiased by collaboration contribution \cite{Hirsch2010a,Stallings2013}, researchers' career stage \cite{Petersen2012}, journal citation skewness \cite{Stringer2008,Petersen2010,Yang2013}, or field size \cite{Radicchi2008,Kaur2013}. The increasing demand for the evaluation and accountability of science both from within the scientific community and the public \cite{Cronin1994a,Kostoff1997, Weingart2005a,Macilwain2010, Lane2011, Fortin2013} means we can no longer afford to rely on flawed indicators of performance.

Citation counts span many orders of magnitude, thus it is ill-advised to work with the raw number of citations directly when creating an indicator \cite{VanNoorden2014}. Furthermore, extensive research on the aging of scientific literature shows that publications' citation rates change over time and eventually reach a steady-state \cite{Egghe2000b,Pollman2000,Burrell2001,Aksnes2003,Bouabid2011,Petersen2014}. As a result of this process of accumulating citations, any set of publications can be characterized by a cumulative distribution of citations. This distribution represents the probability of a publication acquiring a given number of citation after a certain elapsed time period.

A lognormal distribution was one of the first proposed functional forms for the citation distribution \cite{Shockley1957}:
\begin{equation*}
P(n) = \frac{1}{n\sqrt{2 \pi \sigma^2}}\, \exp\left(-\frac{(\ln n - \mu)^2}{2\sigma^2}\right)\;\;,
\end{equation*}
where $\mu$ and $\sigma$ represent, respectively, the mean and standard deviation of $\ln n$. Several researchers have since provided empirical evidence for the use of a lognormal model to study citation distributions \cite{Egghe2001,Redner2005,Radicchi2008,Stringer2008,Radicchi2012,Wang2013a}.

More recently, inspired by Burrell's idea of the existence of a latent variable that determines the number of citations receive by a publication \cite{Burrell2001,Burrell2003a}, Stringer et al.\ used a modified lognormal model to demonstrate that the distribution of the number $n$ of citations to publications published in a given journal in a given year converges to a stationary discrete lognormal functional form after, on average, ten years \cite{Stringer2008,Stringer2010}. With their model, Stringer et al.\ can successfully quantify the long-term impact of publications published in a scientific journal. This suggests that the framework of the discrete lognormal may be used to develop an unbiased bibliometric indicator of scientific impact at several levels.


\section{Objectives}

The primary goal of my research is the quantification of the effect of gender diversity in creative teams. I first present a quantitative analysis of the origins of gender disparities in two distinct domains that are each of paramount importance to society as whole: scientific collaborations, the main drivers of knowledge creation worldwide, and movie-making teams, the creators one of the most popular forms of entertainment. I then focus on the quantification of the impact of the work produced by some of these teams. Namely, I present a framework to quantify the long-term impact of scientific publications.

In \autoref{ch:collaboration}, I study gender diversity in scientific collaborations. Historically, female researchers have been at a disadvantage in STEM disciplines. Females have lower publication rates and shorter careers than males. These observed gender disparities make it difficult to interpret differences in collaborations patterns between male and female researchers. I perform a quantitative analysis of researcher collaborations that properly controls for these historic disadvantages suffered by females. I also analyze systemic differences both between and within several STEM disciplines.

Some researchers have posited that observed gender differences in science may be due to innate ability differences between genders, or females choosing to leave academia. For these reasons, in \autoref{ch:movies}, I turn my studies to the acting career, as it is a profession with no innate differences between males and females but one where gender discrimination nevertheless still exists. I propose a possible cause for the low female representation among actors in the in the U.S. movie industry. I then find how the gender diversity of producers and directors influences the gender composition of actors in a movie production. I also investigate the role of genre and movie budget on female representation in the industry.

To determine the effect of individual characteristics on team dynamics, we need to quantify the impact of the output from those teams. Therefore , in \autoref{ch:lognormal}, I design and rigorously validate a principled framework to measure the long-term impact of scientific publications grounded on the functional form of the discrete lognormal distribution. I use this framework to construct a bibliometric indicator to measure the scientific impact of the publications authored by a researcher and those associated with a given research institution.
