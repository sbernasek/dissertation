\chapter{Probable causes of gender discrimination in the U.S. movie industry}
\label{ch:movies}

The work in this chapter is submitted for publication and was completed with contributions from
Murielle Dunand, and
Lu\'{i}s Amaral.


\section{Abstract}

Gender parity has been slowly but steadily increasing in many sectors of society. One sector where one would expect to see near gender parity is the movie industry, yet the numbers of females in most function of the U.S. movie industry remain surprisingly low. Here, we study the historical trends of female representation among actors, directors, and producers and attempt to gain insights into the causes of the lack of gender parity in the industry. We demonstrate that the advent of the studio system, a period where the ``Big Five'' Hollywood studios deliberately cooperated to control all aspects of the movie industry, had an extremely negative impact on female representation. Indeed, female representation among actors, directors, and producers dropped by more than half after the emergence of the studio system, to values so low that the gender imbalance is still observed presently. Moreover, we find that the gender diversity of a movie's producers influences both the gender of the director and the gender composition of the cast, and that female directors have a statistically significant preference for more gender-balanced casts. Additionally, we find that female directors are over-represented in two genres --- Documentary and Romance --- but under-represented in seven other genres. Lastly, we find that actress representation in higher budget movies grew during the studio system, and that the increase in female representation in the 1960s was most evident in the lowest budget movies.

\clearpage

\section{Introduction}

Gender diversity is increasingly regarded as a desirable condition by educational, business, and governmental organizations. Recent research shows that more gender-balanced groups are better at complex decision-making \cite{Woolley2010} and females show less self-interest are are better at complex moral reasoning than males \cite{Bart2013}. Indeed, the proportion of women faculty members in many STEM fields has been steadily increasing \cite{Duch2012}, as has the number of females in corporate suites and in political office \cite{Zaichkowsky2014,Jalalzai2013}. These trends are positive because the absence of women in leadership positions has a negative impact on women's aspirations and advancement and may perpetuate gender biases \cite{Dean2008}.

A factor muddying the discussion of the causes for lack of gender diversity is the argument that males may be better suited to some professions (i) because of greater physical strength, greater mathematical ability, or some other advantage; or (ii) because, unlike females, they do not have to interrupt their careers due to childbearing. However, there is one career for which neither of these arguments would `hold much water' --- acting. Indeed, unlike many other professions for which it is much easier to name prominent male exponents than female exponents, the same is not true for acting: Marlene Dietrich, Katharine Hepburn, and Meryl Streep are just as recognizable as Douglas Fairbanks, Humphrey Bogart, and Tom Hanks. Moreover, the fact that most actors participate in at most a single movie per year and can go several years without appearing in a motion picture, makes the career more flexible and amenable to actresses taking time away to care for young children.

Yet, there is evidence of significant gender discrimination against females in the U.S. movie industry \cite{Smith2014,DePater2014,Smith2017}. Females not only are offered less roles than males in certain markets \cite{Dean2008}, they will also feature in fewer films if they have repeatedly co-starred with the same counterparts \cite{Lutter2013}, or as they age \cite{Bazzini1997,Lincoln2004}. Indeed, age affects an actor's earnings potential differently depending on gender. For females stars, movie earnings peak in the mid-thirties, whereas for males stars they do not peak until they reach fifty \cite{DePater2014}.

We believe that the study of the historical patterns of female representation among actors is likely to yield insights into the causes of gender discrimination without the confounding effect of \textit{potentially} different innate abilities for the profession. Thus, we study here the temporal evolution of female representation in the cast of over 15 thousand U.S.-produced movies released between 1894 and 2011. We find that prior to the establishment of the Hollywood studio system (1920-1930) \cite{Deutelbaum1989}, female representation stood at nearly 30\%, but that it had decreased by nearly a third by the late 1940s and that it would take another 15 years before it returned to pre-studio system levels. Below, we show that concentration of decision power among a small cadre of male executives predated the drop in female representation and that only the breakdown of the studio system let female representation rise again.

\begin{table}[t]
\caption[Coverage of gender information for the movies in our dataset.]{\textbf{Coverage of gender information for the movies in our dataset}. See Appendix \ref{sec:methods:movies} for details on gender assignment.}
\label{tab:movies:data}
\begin{tabular}{lcrrr}
 ~   & ~              & \multicolumn{3}{c}{\textbf{Gender}} \\
\cmidrule{3-5}
\textbf{Role} & \textbf{Coverage} & \textbf{Males} & \textbf{Females} & \textbf{Unknown} \\
\midrule
Actors    & \actorCoverage{}\%  & \maleActors{}     & \femaleActors{}     & \naActors{}    \\
Directors & \dirCoverage{}\%    & \maleDirectors{}  & \femaleDirectors{}  & \naDirectors{} \\
Producers & \prodCoverage{}\%   & \maleProducers{}  & \femaleProducers{}  & \naProducers{} \\
\end{tabular}
\end{table}


\section{Background}

The early movie industry was considerably more diverse in terms of gender and geography than it would become by the time the Great Depression arrived. Until the mid 1910s, France, Italy and the U.S. were all important movie production countries. Within the U.S., movies were being produced along the East Coast, from New York to Florida. However, within fifteen years, this situation would change dramatically. In the U.S., the attempt by Thomas Edison and the Motion Picture Patent Company (MPPC) to control movie production pushed many in the industry to relocate to California, and away from the legal reach of the MPPC (Fig.~\ref{fig:movies:trend_a}). In Europe, the first World War greatly hindered the development of the industry. As a result, by the 1920s, Hollywood was the dominant player in the global movie industry both in terms of the number of movies being produced (Fig.~\ref{fig:movies:trend_a}) and in terms of the profits captured \cite{Scott2005}.

Economic growth and co-location prompted industry consolidation and the emergence of the so-called ``studio system''. The ``Big Five'' studios (MGM, Paramount, Warner Bros., RKO, and Fox) formed a cartel that controlled every aspect of a motion picture, from the casting of actors, hiring of the director and the screenwriters, all the way to the distribution and exhibition of the final movie \cite{Deutelbaum1989}. Through the studio system, a handful of individuals --- men such as Louis B. Mayer, David Sarnoff, David O. Selznick, or Jack Warner --- gained essentially absolute control over the industry.

\begin{figure}[t]
    \begin{subfigure}{0.7\textwidth}
        \includegraphics[width=\textwidth]{figures/movies_trend}
        \phantomcaption
        \label{fig:movies:trend_a}
    \end{subfigure}
    \begin{subfigure}{0\textwidth}
        \phantomcaption
        \label{fig:movies:trend_b}
    \end{subfigure}
    \begin{subfigure}{0\textwidth}
        \phantomcaption
        \label{fig:movies:trend_c}
    \end{subfigure}
    \begin{subfigure}{0\textwidth}
        \phantomcaption
        \label{fig:movies:trend_d}
    \end{subfigure}
\caption[Historical trends of gender imbalance in the U.S. movie industry.]{\textbf{Historical trends of gender imbalance in the U.S. movie industry}. (\textbf{a}) Timeline of 20th century events relevant to the U.S. movie industry. Orange shadings indicate the rates of TV adoption in U.S. households (from unsaturated to saturated: $<30\%$, $<60\%$, $<90\%$) \cite{Steinberg1980}. Blue bars identify, chronologically, the duration of the MPPC control \cite{Scott2005}, consolidation of the studio system \cite{Deutelbaum1989}, and the blacklisting of industry participants \cite{Buhle2004}. Red bars represent major wars (chronologically, World War I, World War II, Korean war, and Vietnam war). (\textbf{b}) Number of U.S.-produced movies released annually and recorded in IMDb. (\textbf{c}) Percentage of movies directed by females as a function of release year. The data shows a U-shape. Remarkably, the percentage of movies directed by females in the early 1900s (dashed line, approximately \dirEarlyMean{}\%) was only reached again in \dirYearRecover{}, having remained below half of that level for \dirYearsBelow{} years (dash-dotted line). (\textbf{d}) Percentage of female producers in movies (mean $\pm$ standard error). As for directors, the percentage of female producers for movies in the early 1900s (dashed line, approximately \prodEarlyMean{}\%) was only reached again in \prodYearRecover{}, having remained below half of that level for \prodYearsBelow{} years (dash-dotted line).}
\label{fig:movies:trend}
\end{figure}


\section{Results}

Using a dataset \cite{Moreira2017} comprising \totalMovies{} U.S.-produced movies released between 1894 and 2011 (see Table \ref{tab:movies:data} and Appendix \ref{sec:methods:movies} for details), we find that the studio system had a similar impact on female representation among movie directors and producers (Figs.~\ref{fig:movies:trend_c}, \ref{fig:movies:trend_d}). By the 1930s, female representation among producers and directors had dropped to less than half of the levels observed prior to 1920. At the level of producers, the drop was particularly severe for executive producers (Fig.~\ref{fig:movies:producers}). This fact is particularly significant because producers' decisions are so impactful. They are responsible for overseeing a movie's finances, selecting and managing the cast and crew, and are involved in all movie-making facets, from conception to distribution \cite{Lauzen1999,Cattani2013}.

\begin{figure}[t]
    \begin{subfigure}{0.7\textwidth}
        \includegraphics[width=\textwidth]{figures/movies_comparison}
        \phantomcaption
        \label{fig:movies:comparison_a}
    \end{subfigure}
    \begin{subfigure}{0\textwidth}
        \phantomcaption
        \label{fig:movies:comparison_b}
    \end{subfigure}
    \begin{subfigure}{0\textwidth}
        \phantomcaption
        \label{fig:movies:comparison_c}
    \end{subfigure}
    \begin{subfigure}{0\textwidth}
        \phantomcaption
        \label{fig:movies:comparison_d}
    \end{subfigure}
\caption[Power and gender discrimination in the U.S. movie industry.]{\textbf{Power and gender discrimination in the U.S. movie industry}. (\textbf{a}) Lack of gender parity is apparent for three of the most visible functions in the movie industry: producers, directors, and actors. Note the dramatic drop in female representation for these 3 functions starting in 1920. We hypothesize that the power structure within the movie industry contributes to gender discrimination. We test our hypothesis in the following panels using data from movies with a single director. (\textbf{b}) Logistic regression on the probability of a director being female as a function of the percentage of producers that are female. We find a significant correlation (pseudo-$R^2 = \logRegRSqrd{}$, $\beta = \logRegBeta{} \pm \logRegBetaSerr{}$, $p \logRegPval{}$), strongly suggesting that the gender of the producers contributes to explaining the gender of the director. (\textbf{c}) Impact of the gender of the director on the gender representation of actors. We find that, compared to male directors, female directors have a significant preference for a more gender-balanced cast (Mann-Whitney test, $U = \actorDirU{}$, $p \actorDirUPval{}$). (\textbf{d}) Linear regression on the percentage of actors that are female as a function of the percentage of producers that are female. We find a significant increase (slope $ = \actorProdSlope{} \pm \actorProdSlopeSerr{}$, $t = \actorProdT{}$, $p \actorProdPval{}$) in the percentage of female actors cast as the percentage of female producers grows (black line, representing bins left-edge values). Movies with no female producers are binned together ($N = \zeroFProd$); remaining movies ($N = \nonZeroFProd{}$) are divided into 10 equal-sized bins.}
\label{fig:movies:comparison}
\end{figure}


The establishment of the studio system also affected the gender diversity of a movie's cast. Figure \ref{fig:movies:comparison} clearly demonstrates that the emergence of the studio system had a negative impact of female representation within casts. Between 1920 and 1940, we observe a reduction of nearly a third in the percentage of females cast for the typical movie.

The temporal evolution of female representation in three of the most visible functions in the movie industry displays the same overall ``U-shape'' (Fig.~\ref{fig:movies:comparison_a}). In the early years of the U.S. movie industry, female representation is high compared to mid-century levels, between \dirEarlyMean{}\% (for directors) and \actorEarlyMean{}\% (for actors). Interestingly, we find a recovery of female representation starting in the mid-1950s for actors (and in the late 1970s for producers and directors). Not coincidentally, the vise-like grip of the Big Five had started to ease just a few years earlier (Fig.~\ref{fig:movies:trend_a}). First, Olivia de Havilland's 1944 legal victory against Warner Brothers Pictures \cite{DeHaviland1944}, started to free actors from the endless contracts tying them to a studio. Then, in 1948, the U.S. Supreme Court ruled that the structure of the movie industry violated anti-trust laws \cite{US1948}.

These results mirror prior findings for screenwriters. Female screenwriters were highly visible at the start of the movie industry \cite{Smith-Doerr2010}. However, this visibility dramatically decreased with the establishment of the studio system. Only recently have female TV and movie screenwriters started to gain recognition again \cite{Bielby1996,Bielby2009}.

An important difference to acting, however, is that the changes being brought by the studio system for screenwriters could be interpreted as supporting the hypothesis that males are innately better writers, and that the reduction in the representation of female screenwriters was due to increased competition for economically attractive positions. No such argument can be made about innate ability for acting. In the case of acting it is unlikely that competition among individuals with different innate abilities is the mechanism driving the historical patterns of female representation within movie casts.


\subsection{Impact of gender of power brokers}

Because of the decision-making power held by producers, we next investigate whether the gender diversity of the producer affects the gender of the director selected for a movie. To test this hypothesis we first perform a logistic regression on the probability of a director being female as a function of the percentage of producers that are female (Fig.~\ref{fig:movies:comparison_b}). We find a significant correlation (pseudo-$R^2 = \logRegRSqrd{}$, $\beta = \logRegBeta{} \pm \logRegBetaSerr{}$, $p \logRegPval{}$), strongly suggesting that the gender of the producers contributes to explaining why the overwhelming majority of directors are male (Fig.~\ref{fig:movies:comparison_a}. See also \cite{Grugulis2012}).

To further test our hypothesis, we verify whether the gender of the director affects the gender representation of actors. Splitting movies according the gender of the director (Fig.~\ref{fig:movies:comparison_c}) reveals that female directors have a statistically significant preference for more gender-balanced casts (Mann-Whitney test, $U = \actorDirU{}$, $p \actorDirUPval{}$). As a final test, we perform a linear regression on the percentage of actors that are female as a function of the percentage of producers that are female (Fig.~\ref{fig:movies:comparison_d}). We find a significant increase (slope $ = \actorProdSlope{} \pm \actorProdSlopeSerr{}$, $t = \actorProdT{}$, $p \actorProdPval{}$) in the percentage of female actors cast as the percentage of female producers grows, indicating that the gender of the producers also contributes to explaining the gender composition of a movie's cast \cite{Lauzen1999,Smith2016}.


\subsection{Impact of movie genre}

A second movie characteristic that will likely affect female representation is genre. Action, Adventure, or War are all genres typically associated with male characteristics, whereas Romance may be more identifiable with females \cite{Smith2014,Wuhr2017}. Additionally, actors need to consider the genre(s) of the movies they participate in. Novice actors are more likely to be hired in the future if they restrict to the same genre, whereas more established actors have a higher chance to get hired if they diversify the genres of their work \cite{Zuckerman2003}.

In order to investigate the role of genre on female representation, we group movies according to genre (Fig.~\ref{fig:movies:genre}). For clarity, we omit genres with fewer than \popularThreshold{} movies. Note that, in IMDb, movies are usually classified into multiple genres (median 2) which means movies sharing an ``unpopular'' genre may still be considered. To check for female discrimination we compare how many females actually directed movies in a given genre with what would be expected under a genre-unbiased null model (Fig.~\ref{fig:movies:genre_b}). We observe that, while female directors are over-represented in \overGenres{} movies, they are under-represented in seven of the fifteen most popular genres (\underGenres{}). Notably, female directors do not appear to be over or under-represented in the two most common genres, Comedy and Drama. These results confirm the impact of gender preconceptions on hiring decisions. Consistent with all the findings reported, as female director representation decreases, so does the percentage of female actors (Fig.~\ref{fig:movies:genre_c}).


\begin{figure}[t]
    \begin{subfigure}{0.8\textwidth}
        \includegraphics[width=\textwidth]{figures/movies_genre}
        \phantomcaption
        \label{fig:movies:genre_a}
    \end{subfigure}
    \begin{subfigure}{0\textwidth}
        \phantomcaption
        \label{fig:movies:genre_b}
    \end{subfigure}
    \begin{subfigure}{0\textwidth}
        \phantomcaption
        \label{fig:movies:genre_c}
    \end{subfigure}
\caption[Impact of genre on director gender.]{\textbf{Impact of genre on director gender}. (\textbf{a}) Number of movies classified into a given genre. Note that, in IMDb, movies are usually classified into multiple genres (median 2). We omit genres with fewer than \popularThreshold{} movies and consider only movies with a single director. (\textbf{b}) Female directors are over-represented (z-score $ > \zThreshold{}$) in \overGenres{} movies but under-represented (z-score $ < -\zThreshold{}$) in \underGenres{} movies. Observed percentage of movies directed by females is indicated by the blue circles. We calculate 95\% and 99\% confidence intervals (light and dark green bars, respectively) by bootstrapping 1,000 samples the evolution of each genre under a binomial process for selecting a movie's director (see Appendix \ref{sec:methods:movies} for simulation details). (\textbf{c}) Historical percentage of female actors (mean $\pm$ standard error) in movie genres with over-represented (Romance), typical (Comedy), and under-represented (Action) female directors. Note that, as female director representation decreases, so does the percentage of female actors. Data is smoothed over a 3-year rolling window. Black dashed line represents level of gender parity.}
\label{fig:movies:genre}
\end{figure}


\subsection{Impact of movie budget}

As the Hollywood studio system was reeling from the lost legal battles of the 1940s, three major societal changes would force the industry to rethink its strategy. Television, which started entering U.S. households in the late 1940s, had reached over 75\% households by the late 1950s \cite{Steinberg1980}. Simultaneously, the Hollywood Blacklist interrupted the careers of screenwriters, actors, and directors with suspect political views \cite{Buhle2004}. A decade later, the Vietnam War, the Civil Rights movement, and second-wave feminism forced new voices into the movie industry. As a reaction, the big Hollywood studios directed their focus towards big budget movies --- blockbusters --- that would have a better chance of bringing people to the theaters and achieve large profits \cite{Garvin1981,Ravid1999}, and left small budget movies to independent studios \cite{Grugulis2012,Faulkner1987}.

Prompted by these changes, we next investigate the impact of movie budget on female representation within movie casts. We have budget information for nearly \budgetCoverage{}\% (\budgetMovies{}/\totalMovies{}) of the movies in our dataset. We partition these movies by decade, and within each decade partition movies into deciles according to budget. In order to better visualize the impact of movie budget and time on female representation, we calculate deviations from the average female representation for all movies within the specific decade (Fig.~\ref{fig:movies:budget}). Along a column in Fig.~\ref{fig:movies:budget}, positive (negative) values indicate that movies within the budget decile have higher (lower) than average female representation.

In the 1910s, prior to the establishment of the studio system, there is no apparent pattern to the fluctuations in female representation according to movie budget. With the establishment of the studio system, however, we observe that higher than average female representation becomes concentrated in lower budget movies. Remarkably, during the 1930s, 1940s and 1950s, higher than average female representation shifts to increasingly higher budget movies (Fig.~\ref{fig:movies:budget}, leftmost green arrow).

We can understand this shift if we assume that female stars `sold' movies just as well as male stars and that higher budget movies --- which where expected to bring in greater revenues --- would require greater gender balance of their casts. Indeed, while some studies have reported no impact of actors or directors on movie income \cite{Ravid1999,Ainslie2005}, others reported that well-known or recently successful actors, directors, and even producers positively impact movie revenues \cite{Elberse2007,Hadida2010}.

In the 1960s, with the emergence of the independent studios and the greater power of male and female movie stars, we observe a dramatic change in the level of female representation as a function of movie budget. While in the previous decade, higher than average female representation was observed for movies with large budgets, during the 1960s higher than average female representation shifts to the movies with the lowest budgets. Note that this shift is accompanied by an overall increase in female representation (Fig.~\ref{fig:movies:comparison_a}). This means that females entering the industry are entering at the `bottom'.

Strikingly, from the 1960s on, we find a steady increase in the budget size of the movies for which female representation is higher than the average (Fig.~\ref{fig:movies:budget}, rightmost green arrow). Again, this could be understood as the industry being unable to keep successful females stuck in the movies with the lowest budget.


\begin{figure}[t]
\includegraphics[width=0.8\textwidth]{figures/movies_budget}
\caption[Female cast participation as a function of time and movie budget.]{\textbf{Female cast participation as a function of time and movie budget}. In order to highlight possible dependency on budget, we calculate the percentage of female actors in the movies released each decade according to budget decile. We show the difference between the percentage for each cell and the mean percentage of female actors per decade. Top row shows the median inflation-adjusted movie budget (U.S. \$millions) for each decade.}
\label{fig:movies:budget}
\end{figure}


\section{Discussion}

Our study suffers from two limitations. First, it does not capture changes in gender representation among starring actors, as the cast is not always shown in starring order on IMDb. Second, we do not know which percentage of female actors in the early 1900s may have been confined to minor roles or whether that percentage changed over time. Nonetheless, the U-shape of the time-series for all major movie industry functions (Fig.~\ref{fig:movies:comparison_a}) suggests that the studio system resulted in females being systematically excluded from most functions in the industry \cite{Smith-Doerr2010,Bielby1996,Bielby2009}.

Our analysis supports the hypothesis that concentration of power in the hands of a few white males during the heyday of the studio system led to exclusion of other groups. The economic shift in favor of actors after the de Havilland decision \cite{DeHaviland1944} and the power that change brought to some actresses enabled them to later play roles as producers and directors leading to a virtuous cycle of increased female presence in the U.S. movie industry. Such a step could have had its own added benefits, as female directors have a slightly higher chance of directing award-winning movies \cite{Lutter2014}, and collaborations between producers and female actors can have a positive impact on a movie's revenue \cite{Narayan2016}.

Our results are consistent with the broader hypothesis that periods in which an industry grows in importance, with increasing financial rewards, and with greater consolidation may be particular susceptible to dramatic decreases in diversity. Thus, our study adds to the known examples of gender discrimination in such areas as computer science (despite the first programmers being female \cite{Sydell2014}, the discipline became extremely popular among males with the advent of the home PC that was almost exclusively marketed to boys \cite{Margolis2003}) and medicine (by 1900, females struggled to be accepted in medical schools, yet in the previous century they performed almost all medical tasks without training \cite{Morantz-Sanchez1985}). This interpretation is also consistent with the increase in female representation in professions that lost prestige, such as teaching elementary school \cite{Boyle2004}.
