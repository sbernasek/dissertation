%Quantitative analysis methods comprise  

I’m honored to be considered for this award as Northwestern ChBE boasts many outstanding researchers whose achievements merit recognition. I feel my work is distinguished by its creative integration of several disciplines and by the broad relevance of its findings. My research falls under the umbrella of quantitative biology. I combine chemical engineering, computer science, and statistics to provide simple explanations for complex biological phenomena by attaching numbers to processes that are notoriously difficult to measure. 

The bulk of my efforts are focused on deciphering how cells make reliable decisions during development. Cellular decisions to grow, divide, die, or differentiate are controlled by systems of biochemical reactions called regulatory networks. Elucidating the general principles underlying the structure and function of these networks is vital to understanding all developmental processes, as well as the diseases that arise when they fail.

One of my projects revealed a novel mechanism underlying a specific neuronal differentiation decision in the fruit fly eye. Proteins called transcription factors coordinate the timing and execution of differentiation decisions by binding to target genes and modulating their expression. The prevailing belief was that virtually all such decisions are triggered by changes in the absolute concentration of relevant transcription factors. In most cases, these beliefs were based on qualitative observations as it is difficult to quantify transcription factor dynamics in vivo. Using computer vision and statistical modeling techniques, I extracted quantitative measurements of transcription factor dynamics from microscope images of fruit fly eyes collected by my collaborators. We showed that differentiation is driven by dynamic changes in the ratio between two transcription factors, and is agnostic to changes in their absolute concentrations as long as the ratio remains constant. I developed a general model based on the statistical physics of transcription factor DNA binding to show that this phenomenon is a natural consequence of competition between transcription factors for common binding sites. The study adds a new dimension to our understanding of how transcription factors execute cellular decisions, and showcases the importance of quantification in biology. 

Another project addresses the more general question of why many components of regulatory networks appear to serve the same purpose. Networks typically contain multiple negative regulators tasked with attenuating expression of a single transcription factor. Despite serving the same purpose, these redundant regulators are often all essential for normal growth, development, and function of complex organisms. Without them, cells make incorrect decisions and development fails. My collaborators discovered that many essential negative regulators are rendered unnecessary when carbohydrate metabolism is slowed. Their experiments surveyed a broad range of developmental contexts, but offered no insight into the underlying mechanism. I developed a computational framework for probing the molecular behavior responsible for the observed phenomenon. My model suggests the experiments reflect a general principle of dynamic systems; they are more sensitive to perturbation when internal dynamics are fast. In this case, transcription factor activity is more sensitive to changes in regulation when mRNA and protein biosynthesis rates are high. We successfully validated this theory by quantifying transcription factor activity in one of the experimental systems. The findings suggest that redundant negative regulators enable development to proceed more quickly by mitigating erroneous cellular decisions when cells are rapidly metabolizing. As shorter developmental times confer a selective advantage upon organisms, this likely represents a novel evolutionary driving force for increased redundancy in regulatory networks. 

Beyond their biological insights, my projects have spawned computational tools that will likely prove valuable to the broader community. My gene network simulation package has already been adopted by two other researchers at Northwestern. I also plan to distribute my transcription factor binding model and computer vision methods, as these resources are broadly applicable to many different biological contexts. By sharing them I hope to promote the adoption of quantitative methods in biology and continue to embrace the spirit of interdisciplinary collaboration that lured me toward this department.





ABSTRACT

Metabolic conditions affect the developmental tempo of most animal species. Consequently, developmental gene regulatory networks (GRNs) must faithfully adjust their dynamics to a variable time scale. We find evidence that layered weak repression of genes provides the necessary coupling between GRN output and cellular metabolism. Using a mathematical model that replicates such a scenario, we find that lowering metabolism corrects developmental errors that otherwise occur when different layers of repression are lost. Through mutant analysis, we show that gene expression dynamics are unaffected by loss of repressors, but only when cellular metabolism is reduced. We further show that when metabolism is lowered, formation of a variety of sensory organs in Drosophila is normal despite loss of individual repressors of transcription, mRNA stability, and protein stability. We demonstrate the universality of this phenomenon by experimentally eliminating the entire microRNA family of repressors, and find that all microRNAs are rendered unnecessary when metabolism is reduced. Thus, layered weak repression provides robustness through error frequency suppression, and may provide an evolutionary route to a shorter reproductive cycle.
