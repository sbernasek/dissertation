Organismal development depends upon countless cell decisions to adopt particular fates at the appropriate time and place. These decisions are executed by systems of biochemical reactions called regulatory networks. Elucidating the general principles underlying the structure and function of these networks is vital to understanding all developmental processes, as well as the diseases that arise when they fail. 

Prior studies of regulatory networks, and the decisions they implement, have heavily relied upon qualitative analysis of experimental data. It has since become clear that quantitative strategies are needed to unravel the complexities of systems-level behavior. The research enclosed in this dissertation therefore combines chemical engineering, computer science, statistics, and experimental data to quantitatively explore how regulatory networks reliably coordinate cell fate decisions.

The findings are consolidated into three distinct chapters. The first two are anchored to a common model system of the \textit{Drosophila} larval eye. They deploy an assortment of novel computational tools, mathematical models, and statistical methods to derive meaningful insight from experimental measurements of the processes that govern cell fate decisions during retinal patterning. The final chapter introduces a mathematical modeling framework to lead the development of an exciting new hypothesis; auxiliary negative regulators enable development to proceed more quickly by mitigating erroneous cell fate decisions when cells are rapidly metabolizing.

Beyond their insights into the mechanics of cell fate decisions, these efforts have spawned several computational tools that may prove valuable to the broader community. All of these resources have been made freely available (see Appendix \ref{appendix:software}), with the hope that their continued development will contribute toward a more quantitative future for developmental biology.
