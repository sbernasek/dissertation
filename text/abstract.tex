Organismal development depends upon countless cell decisions to adopt particular fates at the appropriate time and place. These decisions are executed by systems of biochemical reactions called regulatory networks. Elucidating the general principles underlying the structure and function of these networks is vital to understanding all developmental processes, as well as the diseases that arise when they fail. 

Prior studies of regulatory networks, and the decisions they implement, heavily relied upon qualitative analysis of experimental data. It has since become clear that the complexities of systems-level behavior necessitate more quantitative strategies. The research enclosed in this dissertation combines chemical engineering, computer science, statistics, and experimental data to quantitatively explore how regulatory networks reliably coordinate cell fate decisions.

Chapter \ref{ch:clones} introduces a computational framework for automated quantitative analysis of genetic mosaics; a class of experiments designed to probe cell fate decisions \textit{in vivo}. The framework combines computer vision and statistics to measure protein levels in individual cells and infer their respective genotypes. It thereby facilitates systematic comparison of cells subject to control and perturbation conditions. The accompanying open-source software eliminates each of the labor-intensive steps of a quantitative workflow, extending the accessibility of quantitative analysis to the broader research community.

Chapter \ref{ch:ratio} explores a novel cell fate decision mechanism underlying photoreceptor specification in the larval fruit fly eye. Computer vision techniques are used to extract quantitative measurements of transcription factor dynamics from a wealth of confocal microscopy data. Statistical analysis of these data reveals that differentiation is driven by dynamic changes in the ratio between two transcription factors, and is agnostic to changes in their absolute concentrations as long as the ratio remains constant. A general model based on the statistical physics of transcription factor DNA binding shows that this phenomenon is a natural consequence of competition between the two transcription factors for common binding sites. The findings add a new dimension to our understanding of how transcription factors coordinate cell fate decisions, and exemplifies the importance of both quantitative and dynamic measurements for characterizing developmental systems. 

Chapter \ref{ch:metabolism} proposes a new theory to explain why the regulatory networks that coordinate cell fate decisions often contain several repressors tasked with attenuating expression of a single target gene. The theory posits that auxiliary negative regulators enable development to proceed more quickly by mitigating erroneous cell fate decisions when cells are rapidly metabolizing. It is supported by a robust collection of qualitative experiments showing that a broad variety of repressor loss-of-function phenotypes are reversed when biosynthesis rates are artificially slowed. A quantitative modeling framework is used explore the mechanistic origin of this effect. Namely, that auxiliary repressors help avert erroneous decisions by expanding cells capacity to buffer excess protein expression. Quantitative measurements of transcription factor activity serve to validate the predictions made by the modeling framework. As shorter developmental times confer a selective advantage upon organisms, these findings may represent a novel evolutionary driving force for increased robustness of cell fate decisions.

Beyond their insights into the mechanics of cell fate decisions, these efforts have spawned several computational tools that may prove valuable to the broader community (see \ref{appendix:software}). All such resources have been made freely available, and their continued development will help advance the field of quantitative biology.