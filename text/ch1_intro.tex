

\textbf{ABSTRACT}

Cells must reliably respond to changes in transcription factor levels in order to execute cell state transitions in the correct time and place. These transitions are typically thought to be triggered by changes in the absolute nuclear concentrations of relevant transcription factors. We have identified a developmental context in which cell fate transitions depend on changes in the relative concentrations of two transcription factors. Here, we quantify the \emph{in vivo} expression dynamics of Yan and Pointed, two essential E-twenty-six (ETS) proteins that regulate transcription during eye development in \emph{Drosophila}. These two factors exert opposing influences; one impedes transcription of gene targets required for differentiation while the other promotes it. We show that both proteins are transiently co-expressed in eye progenitor cells and also during photoreceptor specification. To decide whether to undergo state transitions, cells respond to the ratio of the two protein concentrations rather than to changes in the absolute abundance of either transcription factor. We show that a simple model based on the statistical physics of protein-DNA binding illustrates how this ratiometric sensing of transcription factor concentrations could occur. Gene dosage experiments reveal that progenitor cells stabilize the ratio against fluctuations in the absolute concentration of either protein. We further show that signaling inputs via the Notch and Receptor Tyrosine Kinase (RTK) pathways set the ratio in progenitor cells, priming them for either transit to differentiation or for continued multipotency. A sustained change in the ratio accompanies the transit to differentiation This novel mechanism allows for distributed control of developmental transitions by multiple transcription factors, making the system robust to fluctuating genetic or environmental conditions.


