\chapter{Conclusion}
\label{ch:conclusion}

In this dissertation, I have presented a rigorous analysis of gender disparities in creative teams. I first analyzed the differences in collaboration patterns between male and female STEM researchers. Then, I studied the origins of gender discrimination in the U.S. movie industry. I have also presented a framework to quantify scientific impact of individual researchers and academic institutions.

My work is noteworthy in that all results are derived from rigorous statistical analysis of large-scale datasets. I used the Web of Science database of scientific publications when quantifying the effect of gender diversity in scientific collaborations and the impact of scientific publications; and the Internet Movie Database when quantifying the effect of gender diversity in movie productions.

In \autoref{ch:collaboration}, I studied gender differences in scientific collaborations. I first proved that, even though female researchers have less distinct collaborators, this is only due to the fact that females publish less than males and have shorter career lengths. I then showed that, despite these disadvantages, females actually have a higher propensity to engage in novel collaborations, suggesting their work to be of higher impact than that of males. Finally, I presented evidence of female exclusion from genomics, a sub-disciplines of molecular biology.

In \autoref{ch:movies}, I present evidence for how females have been discriminated against in the U.S. movie industry. Namely, I demonstrated that during the years of the Hollywood studio system, female representation among actors, directors, and producers dropped by more than half. This under-representation may be at least partially responsible for today's observed gender imbalance in the movie industry as I also found that the gender diversity of a movie's producers influences both the gender of the director and the gender composition of the cast, and that female directors have a statistically significant preference for more gender-balanced casts. Additionally, I showed that female directors are over-represented in \overGenres{}, and under-represented in seven other genres \underGenres{}. Finally, I found that higher than average female representation became concentrated in higher budget movies during the studio system, but in the 1960s higher than average female representation shifted to the movies with the lowest budgets.

In \autoref{ch:lognormal}, I put forth the notion that scientific publications have a latent ``citability'' that can be estimated using the asymptotic number of citations. Specifically, I determined that the asymptotic number of citations $n_a$ to sets of publications by a researcher or associated with an academic department can be described by a discrete lognormal distribution. I performed a principled statistical analysis of the properties of this distribution and showed that the mean citability, $\mu$, can be used as an unbiased bibliometric indicator of scientific impact for researchers, departments, and journals. Furthermore, $\mu$ is resistant to manipulation, unlike other popular indicators such as the \emph{h}-index, and can be well approximated by the median of the logarithm of $n_a$.
