\chapter{Conclusions}
\label{ch:conclusion}


\section{ Tools for a quantitative future }

- Need for quantification in each set of experiments
- Danger of 'feedback' diagrams
- Need for quantitative measurements


\section{ Development viewed through the lens of control theory }

- Value of 'dynamics' only perspective
- Discussion on link to molecular mechanisms
- Ratio control circuit


\section{ A novel driving force in the evolution of GRN topologies }

- Cartoon analogy
- Relation to temperature - would require knowledge of temperature/metabolism scaling
- Introduction of simpler model
- Exploration of tradeoff between synthesis and degradation
- Potential for a complete theoretical framework
- Is redundant regulation the only link? What about promoters?



% recent papers on metabolism & phenotypes --- see PNAS presentation and paper to rich/luis
% recent gap genes example, all information encoded in a handful of factors
% emphasis on dynamics - simple model, how does one process scale versus another



%\section{Dynamic perspective of cell behavior}
%	\subsection{Dynamics matter}
%		\paragraph{Decisions are localized in time.}
%	\subsection{Control theoretical insights into cell behavior}
%		\paragraph{Yeast papers}
%		\paragraph{Khammash papers}


\Chapter{Conclusions}

reconcile pnt/yan mechanism with yan over-expression

discuss impact of clones pipeline on pnt/yan study


- development of 'penetrance theory' based on analytical framework using moment equations
- rigorous exploration of other control architectures
- quantitative characterization of a well known system (yeast)
-


\section{Robustness as an evolutionary design principle}

	\subsection{Ubiquity of noise}	
	
	\subsection{Scale free ratiometric sensing}
	
	\subsection{Selective pressure on negative regulation}
	
		\paragraph{Evidence of further phenotype suppression (papers sent to Rich)}

\section{Harnessing the power of quantitative models}

	\subsection{Demand for accurate measurements}
	
	\subsection{Knowledge at the expense of molecular detail}
		\paragraph{Dynamic model of ratiometric control.}		
		
	\subsection{Generating hypotheses to guide experiments}
		\paragraph{Cell decisions under variable temperature.}		
		\paragraph{Robustness to loss of promoters.}
		\paragraph{Exploration of simpler systems (yeast mating response).}
		
\section{Outlook for studies of cell decisions}

	\subsection{Open questions for further exploration}
		\paragraph{Robustness of cell decisions to other environmental conditions.}	
					
	\subsection{Consolidating knowledge through collaboration}
	
		\paragraph{Grounding theory with experimental validation.}		
		\paragraph{Open source ecosystem.}




---------- GET CITES FROM NRSA

Robustness is a fundamental organizational principle underlying the evolution of biological systems7. This need for reliability amidst genetic and environmental perturbations is typically assumed to explain the prevalence of redundancy and feedback within gene regulatory networks8. The proposed research poses a complementary explanation for the ubiquity of these regulatory features; redundant negative feedback confers a selective advantage by enabling rapid growth and development. By exploring this hypothesis, the study promises to identify a novel driving force for increased complexity in gene regulatory network topologies. 



Loss of function mutations affecting regulatory mechanisms have been implicated in the emergence of cancer, autoimmune disease, neurological disorders, diabetes, cardiovascular disease, and morphological abnormalities. Understanding when and how regulation fails is vital to controlling its proper function in humans.

1.	Misregulation of gene expression results in human cancer and disease. Genome-wide association studies have shown non-coding regulatory DNA variation to be strongly associated with human cancer and disease1. Understanding how regulatory variants give rise to disease states is vital to the development and implementation of novel medical strategies. These functional relationships have been explored for a number of transcriptional, post-transcriptional, and epigenetic regulators, but in most cases the precise molecular details remain elusive2–5. This study proposes both mechanistic and phenomenological frameworks for studying the emergence of developmental errors when regulation fails to attenuate transient signals. By studying how developmental processes fail when regulation is perturbed in Drosophila and yeast, the proposed research will advance our understanding of how regulation ensures accurate and timely cellular decisions in all animals.  

In the future, model-informed design of engineered repressors could serve as a platform for tuning signaling dynamics in a manner conducive to proper human development and health.

The prevalence of redundant regulation in gene regulatory networks is often ascribed to a need for robustness against genetic and environmental variability8. These experiments motivate an alternate hypothesis; redundant negative feedback facilitates faster growth and development. While intriguing, the data are limited to fractions of animal populations exhibiting abnormal phenotypes and do not address the underlying mechanism. Any approach toward understanding this phenomenon must bridge the gap between gene expression and phenotypic outcomes with sufficient generality to explain the breadth of surveyed systems. The proposed research plan provides an extension to this hypothesis; redundant negative feedback facilitates faster growth and development by coupling developmental gene expression programs to physiological cell state. My hypothesis is predicated on the notion that cells make mistakes when global rates of transcription and translation outpace regulatory networks’ abilities to attenuate transient signals. From this perspective, redundant repressors provide overflow capacity to buffer against increased protein synthesis, ensuring the timely degradation of transiently expressed proteins when growth and development are fast. Together, increased biosynthesis and sufficient negative feedback would enable development to proceed more quickly without incurring additional errors.

The breadth of experiments in Drosophila points toward a dynamic phenomenon agnostic to the molecular detail of repressors and their targets. To demonstrate the generality of my hypothesis, I propose a coarse-grained framework relating expression of a generic protein to population-wide phenotype distributions that does not depend upon the specifics of each developmental system. My goal is to predict how removal of a negative feedback mechanism causes protein levels within affected cells to differ from their unperturbed counterparts. As nuanced changes in expression yield abnormal morphologies in vivo, differences in protein levels manifest as differences in phenotype penetrance25. I aim to quantify both how different a population of cells behaves when negative feedback is removed and how the magnitude of this difference depends upon biosynthesis rates.

The framework suggests the experiments reflect a general principle of dynamic systems; they are more sensitive to perturbation when internal dynamics are fast. In this case, transcription factor activity is more sensitive to changes in regulation when mRNA and protein biosynthesis rates are high.






Together, these results indicate that tightly coordinated competition between Yan and Pnt lies at the heart of neuronal fate commitment. These observations are consistent with the notion that Yan and Pnt compete for occupancy of shared binding sites in the promoter region of downstream effectors of neuronal differentiation(Gabay et al., 1996; O’Neill et al., 1994). Consequently, altering the expression dynamics of either protein is expected to increase the frequency of erroneous fate commitment, and subsequently increase the likelihood of a roughened eye phenotype. 