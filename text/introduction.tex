

\textbf{ABSTRACT}

Cells must reliably respond to changes in transcription factor levels in order to execute cell state transitions in the correct time and place. These transitions are typically thought to be triggered by changes in the absolute nuclear concentrations of relevant transcription factors. We have identified a developmental context in which cell fate transitions depend on changes in the relative concentrations of two transcription factors. Here, we quantify the \emph{in vivo} expression dynamics of Yan and Pointed, two essential E-twenty-six (ETS) proteins that regulate transcription during eye development in \emph{Drosophila}. These two factors exert opposing influences; one impedes transcription of gene targets required for differentiation while the other promotes it. We show that both proteins are transiently co-expressed in eye progenitor cells and also during photoreceptor specification. To decide whether to undergo state transitions, cells respond to the ratio of the two protein concentrations rather than to changes in the absolute abundance of either transcription factor. We show that a simple model based on the statistical physics of protein-DNA binding illustrates how this ratiometric sensing of transcription factor concentrations could occur. Gene dosage experiments reveal that progenitor cells stabilize the ratio against fluctuations in the absolute concentration of either protein. We further show that signaling inputs via the Notch and Receptor Tyrosine Kinase (RTK) pathways set the ratio in progenitor cells, priming them for either transit to differentiation or for continued multipotency. A sustained change in the ratio accompanies the transit to differentiation This novel mechanism allows for distributed control of developmental transitions by multiple transcription factors, making the system robust to fluctuating genetic or environmental conditions.


% recent papers on metabolism & phenotypes --- see PNAS presentation and paper to rich/luis
% recent gap genes example, all information encoded in a handful of factors
% emphasis on dynamics - simple model, how does one process scale versus another


\Chapter{Introduction}

\section{Minor decisions with major consequences}

	\subsection{Reliability of developmental outcomes}	
	\subsection{Molecular origins of disease}

\section{Quantitative analysis of cell decisions}

	\subsection{The need for quantitative measurements}	
	\subsection{Progressive shift toward quantification}
	\subsection{Modern approaches toward measuring cellular processes}
		\paragraph{Advanced microscopy and high throughput analysis}
	
\section{Modeling cell decision making}

	\subsection{Deterministic models of developmental patterning}	
	\subsection{Cell decisions in stochastic environments}

\section{Dynamic perspective of cell behavior}

	\subsection{Dynamics matter}
		\paragraph{Decisions are localized in time.}
	\subsection{Control theoretical insights into cell behavior}
		\paragraph{Yeast papers}
		\paragraph{Khammash papers}


\Chapter{Conclusions}

reconcile pnt/yan mechanism with yan over-expression

discuss impact of clones pipeline on pnt/yan study


- development of 'penetrance theory' based on analytical framework using moment equations
- rigorous exploration of other control architectures
- quantitative characterization of a well known system (yeast)
- 


\section{Robustness as an evolutionary design principle}

	\subsection{Ubiquity of noise}	
	
	\subsection{Scale free ratiometric sensing}
	
	\subsection{Selective pressure on negative regulation}
	
		\paragraph{Evidence of further phenotype suppression (papers sent to Rich)}

\section{Harnessing the power of quantitative models}

	\subsection{Demand for accurate measurements}
	
	\subsection{Knowledge at the expense of molecular detail}
		\paragraph{Dynamic model of ratiometric control.}		
		
	\subsection{Generating hypotheses to guide experiments}
		\paragraph{Cell decisions under variable temperature.}		
		\paragraph{Robustness to loss of promoters.}
		\paragraph{Exploration of simpler systems (yeast mating response).}
		
\section{Outlook for studies of cell decisions}

	\subsection{Open questions for further exploration}
		\paragraph{Robustness of cell decisions to other environmental conditions.}	
					
	\subsection{Consolidating knowledge through collaboration}
	
		\paragraph{Grounding theory with experimental validation.}		
		\paragraph{Open source ecosystem.}