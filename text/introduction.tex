\chapter{Introduction}

The natural world presents a stunning variety of multicellular organisms. We tend to distinguish them by the physiological traits that define their stereotyped identities. For instance, we recognize penguins by their unusual stature, black and white fur, long beak, and impressive ability to wobble around on ice. These features are a culmination of many complex cellular processes collectively known as \emph{development}.

Development begins with a single cell, with subsequent growth and division events driving progression toward adulthood. Cells acquire increasingly specific roles and functions as growth proceeds, ultimately giving rise to stereotyped adult morphologies. This process, known as lineage restriction, demands that each cell decides to adopt the correct fate at the appropriate time and place.

Cell fate decisions are remarkably robust, yielding consistent phenotypes amidst the vast array of conditions encountered in natural environments. They are so reliable that we often take them for granted. For instance, it is difficult to imagine a scenario in which we might mistake another human for a penguin. However, they can and do exhibit minor variation that can have dire consequences for human health. Researchers therefore continue to study these complex processes in the hope that they might one day be able to control their behavior; either to exploit them in novel biotechnologies, or rectify the diseases that arise when they fail. Research efforts are predominantly motivated by two fundamental questions. First, how do cells make decisions? Second, how do they make them reliably? 

This dissertation addresses subtle aspects of both questions by combining chemical engineering, computer vision, and statistics to extract meaningful insight from experimental data. The remaining sections of this chapter serve to prime the reader with the context needed to situate the findings within the broader literature. 

\section{Molecular origins of cell fate decisions}

Early experiments in a variety of model organisms demonstrated that cell fate decisions are encoded in the genome. Genetics were therefore believed to provide a predefined road map for the tortuous journey from embryo to adulthood, inspiring researchers throughout the twentieth century to probe the roles of individual genes in coordinating adult phenotypes. Most of their efforts embraced a common philosophy; break something and see what happens. Nowhere is the legacy of this approach more pronounced than in the common fruit fly, \textit{Drosophila melanogaster}, whose genes are predominantly named after the phenotypes that emerge in their absence. Removing \textit{eyeless}, \textit{wingless}, or \textit{hairy} may now seem rudimentary, but these genetic perturbations, and others like them, were vital to the discovery of genetic components and architectures that dictate cell fate decisions in all organisms. They revealed that some genes confer pleiotropic functions across several stages of development, while others are limited to a single context. They also showed that some genes are only essential for proper development when others are absent, rendering them redundant under normal conditions. Furthermore, \emph{Drosophila} continues to be a prominent model system for studying developmental processes today, owing to its conveniently short life cycle, wealth of prior knowledge, and deep library of available genetic machinery.

Gene-specific reporters have augmented traditional genetic perturbations by providing localized readouts of transcript and protein abundance. Researchers can now break something and see what happens to specific components of the developmental program. Reporters led to the discovery of transcription factors; proteins that bind the promoter region of other genes in order to modulate their expression. Multiple transcription factors may interact with each other, allowing for the assembly of gene regulatory networks (GRN) that integrate upstream signaling cues to elicit specific changes in gene expression. 

GRNs thus arm cells with a chemical mechanism to orchestrate cell fate decisions in space and time. A prominent example occurs during the during the earliest stages of \textit{Drosophila} embryogenesis, where spatial morphogen gradients drive variegated expression of four Gap genes. The expressed proteins trigger subsequent developmental events in a concentration-dependent manner, inducing localized cascades of GRN activity that ultimately give rise to distinct organs. 

The spatial resolution of GRN activity is thought to be refined over time. A prominent example of this phenomenon occurs during retinal patterning of the \textit{Drosophila} eye. The larval eye is a particularly popular setting for studies of cell fate decisions because the temporal history of cell fate decisions is encoded in static images of the eye imaginal disc \cite{Pelaez2015}. During the third larval instar, a wave of differentiation steadily progresses across a disordered pool of multipotent cells \cite{Ready1976,Tomlinson1987}. Differentiating cells propagate this morphogenetic furrow (MF) by relaying extracellular cues downstream \cite{Lubensky2011}. How cells interpret these signals and commit to a particular fate remains an open question for all neuronal types. However, experiments have revealed that R8 photoreceptor neurons are recruited from pools of adjacent cells at the leading edge of the furrow. The first cell to differentiate uses paracrine signaling to laterally inhibit cell fate transitions among its immediate neighbors, resulting in the repeated pattern of regularly-spaced R cells that give rise to the crystalline lattice structure of the adult eye. It transforms a disordered field of cells into an ordered template for subsequent stages of patterning, thus refining the spatial resolution of the developing eye field.

The canonical model suggests that the initial race to an R8 fate is non-deterministic. Experiments show the R8 cell fate transition is driven by the expression of Atonal, a transcription factor initially induced in all cells along the leading edge of the furrow. Minor fluctuations in cells precise spatial arrangement lead to variable reception of inductive cues \cite{}, yielding stochastic variation in both Atonal levels and, consequently, cells propensity to adopt an R8 fate. The eventual R8 cell therefore does not appear to be predetermined.

This example demonstrates another core feature of GRNs: their components exhibit remarkable heterogeneity, even within the same developmental context. Variation may be attributed to both intrinsic and extrinsic sources of noise, each of which can manifest in the population-wide penetrance of adult phenotypes. Recent evidence further suggests that some systems may even leverage heterogeneity, using GRNs to amplify and reinforce stochastic fluctuations to limit signal responses to a random subset of cells. Pel'{a}ez et al proposed a similar mechanism to explain why subsequent R cell fate transitions coincide with rapid increases in transcription factor expression heterogeneity \cite{Pelaez2015}.

These experimental observations and others like them form our contemporary systems-level view of development, in which complex networks of regulatory interactions guide cells toward the appropriate fates by refining their stochastic behavior over time. However, it is unknown precisely how cell fates are resolved from simultaneous activities of multiple transcription factors. It is also unclear how cells integrate stochastic inputs to execute reliable cell fate decisions. Furthermore, new evidence challenges the notion that cell fates are spatially resolved as development proceeds. A recent study combined quantitative measurements and information theory to show that the Gap proteins provide sufficient spatial context to uniquely define the fates of all cells. The authors further argued that downstream GRNs likely decode these signals with near-optimal efficiency, suggesting that some cell fates may actually be determined during the earliest stages of embryogenesis. These ambiguities persist despite the wealth of experimental data generated throughout the past century. It is therefore clear that new approaches are needed to unravel the complexities of GRNs that govern cell fate decisions.

\section{The need for quantitative analysis}

While experimental perturbations continue to supply some of the most potent tools in the arsenal available to researchers, it has become clear that qualitative readouts of phenotypes and reporter levels offer inadequate insight to explain the complex interactions that govern cell fate decisions. Moreover, attempts to rectify faulty decision mechanisms or engineer new ones, such as for cancer treatment or the generation of induced pluripotent stem cells, demand predictive models backed by quantitative data.

The focus has therefore gradually shifted from break something and \textit{see} what happens to break something and \textit{measure} what happens. The transition is supported by simultaneous advances in the resolution with which we can quantify the abundance of gene products during development. The state of the art has progressed from aggregate measurements of transcript and protein content across entire tissues, to \emph{in vivo} measurements of individual cells, to \emph{in situ} quantification of single-cell expression dynamics. These advances bestow mathematical models with the quantitative insight required to generate testable predictions for further experimental validation.

Unfortunately, quantitative analysis strategies have not been universally adopted by the research community, leading to the sustained prevalence of subjective analysis in the literature. One plausible explanation is that quantification often demands computational proficiency that falls beyond the scope of many experimental labs. Interdisciplinary collaborations are becoming more frequent and should help alleviate this challenge, but they do not provide a permanent solution. These studies could benefit from the introduction of open-source automated analysis software platforms analogous to those that have revolutionized other subdisciplines of biology. For example, without the support of automated sequence alignment software, RNA-seq would be inaccessible to all but a few researchers with extensive programming and statistical modeling experience. Similar tools are therefore needed to support quantitative measurements of gene expression during development, thus lowering the barrier to adoption of data-driven modeling of cell fate decisions.

\section{Quantitative analysis of GRN behavior}

Quantitative measurements and mathematical models have proven invaluable to studies of GRN behavior in a wide variety of biological systems. Most modeling efforts fall into one of two categories; those that recapitulate molecular mechanism, and those that empirically capture systems-level behavior. 

The first class of models strive to parameterize specific biomolecular interactions by fitting a model directly to data. They typically describe the time-evolution of transcripts and proteins using systems of coupled ordinary differential equations (ODEs) reminiscent of those familiar to chemical engineers and ecologists. Despite the illusion of mechanistic detail, these models still deploy a healthy dose of abstraction. None of the commonly used rate represent true elementary reactions, instead opting for empirical representations such as linear degradation and cooperative binding kinetics. Nevertheless, many novel GRN behaviors and functions have been elegantly proposed and tested in this manner \cite{Yu2008,Paulsen2011}. 

The second class of models forego molecular detail in favor of a coarse-grained representation of a particular phenomenon. These approaches provide a powerful means to identify, characterize, and predict behaviors that span a broad variety of biological contexts. Among the common modeling frameworks, control theory has proven particularly fertile for generating and testing hypotheses related to GRN dynamics. Bacterial chemotaxis offers a compelling example in which a history of experimentally-inspired molecular models were supplanted by a simple integral control framework. Analogous strategies have drawn inspiration from various disciplines to further our understanding of threshold response \cite{Melen2005,Graham2010}, signal transduction \cite{Benziger2018}, fold-change detection \cite{Adler2018}, and many other functions of GRNs.




\section{Modeling cell fate decisions in \emph{Drosophila}}

Quantitative analysis has similarly reinvigorated the study of cell fate decisions, with many models having sought to explain how cell fates are reliably resolved from the spatiotemporal signaling cues encoded in GRN activity. Melen et al used a simple system of ODEs to explore how cells generate all-or-none responses to morphogen gradients in the \textit{Drosophila} ventral ectoderm \cite{Melen2005}. They proposed that an ultrasensitive response mechanism dictates the expression of Yan, a transcriptional repressor responsible for impeding cell fate transitions. Graham et al later used a different system of ODEs to argue that Yan plays a different role in the larval eye, instead forming a bi-stable switch by antagonizing a transcriptional activator named Pointed (Pnt) \cite{Graham2010}. Shwartz et al refined the model to include autoregulatory interactions between each the Pnt isoforms that drive the sustained induction of Pnt needed to flip the switch \cite{Shwartz2013}. Pelaez et al then published quantitative measurements of Yan expression that contradict all of these models \cite{Pelaez2015}. 


Prior experimental studies identified two ETS-domain transcription factors, Yan and Pointed (Pnt), as key regulators of differentiation \cite{Lai1992,Rebay1995}. Yan and Pnt repress and activate transcription, respectively. Until recently, the two proteins were predominantly believed to comprise a bistable switch enforced by mutual inhibition(Graham et al., 2010). My collaborators recently showed that Yan and Pnt are co-expressed during eye development, and Yan exhibits transient dynamic behavior(Boisclair Lachance et al., 2014; Peláez et al., 2015). These observations refute the existing model, motivating further investigation of the Yan and Pnt network. 




\section{Roles for negative feedback in GRNs}
 
Negative feedback mechanisms serve many varied roles in developmental systems \cite{Freeman2000}. From an engineering perspective, negative feedback is obviously useful for maintaining a particular cell state – that is, rejecting exogenous disturbances and driving cells toward a desired set point. Computational studies have proposed numerous potential uses for negative feedback in biological systems, including adaptation to sustained disturbances \cite{Alon2007,Behar2007,Yi2000}. Advanced genetic tools and gene expression assays with high temporal and cellular resolution continue to demonstrate each of these diverse functions of negative feedback in vivo.

In human engineered electrical systems, proportional feedback is known to linearize input-output relationships, thus improving information transmission by expanding dynamic range \cite{Black}. The same behavior has been reported in developmental signaling cascades \cite{Bhalla2002,Cheong2011,Paulsen2011,Yi2003,Yu2008}. Yi et al. used a FRET-based reporter of the G\textalpha-subunit to demonstrate dose-response alignment of G-protein signaling activity at the highest level of the yeast pheromone response signaling cascade \cite{Yi2003}. Yu et al. quantified pFus3 activity following pathway activation to demonstrate that pFus3-mediated negative feedback further increases signaling fidelity through dose-response alignment of subsequent stages of the pathway \cite{Yu2008}. Paulsen et al. showed that a synexpressed negative regulator expands the dynamic range of BMP4 signaling in HEK293 cells \cite{Paulsen2011}. The authors also demonstrated that feedback suppresses transduction of correlated extrinsic noise, further improving signal fidelity. Notably, in all of these cases computational models were used to demonstrate the dynamic principles underlying each of the experimental observations. 

Paulsen et al. also report increased variation in Xenopus gene expression and morphology upon elimination of synexpressed negative feedback, suggesting that negative feedback can suppress phenotypic variation. Experiments in \textit{Drosophila} point toward a similar role for microRNAs in buffering developmental processes against both environmental and genetic variability \cite{Cassidy2016,Cassidy2013,Li2009}. Li et al. qualitatively investigated the effects of temperature fluctuations on sensory organ development under perturbations to miR-7 activity \cite{Li2009}. The authors concluded that miR-7 stabilizes both gene expression and fate commitment decisions against environmental fluctuations during sensory organ development. Using directional selection to quantify the heritability of increased scutellar bristle formation, Cassidy et al. demonstrated that miR-9a canalizes development by suppressing the effects of genetic variants that promote bristle formation \cite{Cassidy2013}. The same group later showed that miR-9a does not universally suppress the penetrance of genetic variants, as no measurable increase in heritability of decreased bristle formation was detected in the absence of miR-9a function \cite{Cassidy2016}. The authors did, however, conclude that miR-9 directly suppresses the phenotypic penetrance of genetic variants deleterious to organismal fitness by measuring viability at elevated temperatures. Together, these studies implicate negative feedback mechanisms, particularly microRNAs in \textit{Drosophila}, in promoting canalization. It remains unclear, however, precisely how this effect is achieved.




\section{Evolutionary drivers of robust cell fate decisions}

Studies of developmental robustness are often traced back to C.H. Waddington’s claim that “there is scarcely a mutant that is comparable in constancy with the wild type” \cite{Waddington1942}. His comment reflects the consistency of adult phenotypes amidst modest levels of genetic and environmental variation, which gives way to considerable variability when severe perturbations are introduced \cite{Bateman1959,Rendel1959,Rendel1966,Scharloo1991}. Robustness has since been attributed to the interactions of genes and their products \cite{Dun1958,Hogness1996,Rutherford1998}, as well as the systems-level architectures that they comprise \cite{Rutherford1998,Paulsen2011,Li2009,Eldar2002,Denby2012,Cassidy2013,Cassidy2016}. A handful of conserved regulatory motifs, and the strategies they implement, are now understood to pervade most developmental processes \cite{Freeman2000,Hartman2001,Alon2007,Marciano2014}. Robustness has thus come to be accepted as a fundamental organizational principle underlying the evolution of all biological systems \cite{Kitano2004}. That is, selection for genotypes that confer robustness is widely assumed to shape the evolution of gene regulatory network topologies. 

The specific evolutionary driving forces for increased robustness of GRNs remain unclear \cite{Siegal2014}. Citing phenomenological examples, Waddington attributed robustness to evolutionary selection for optimal traits. More recently, Siegal et al. argued that developmental robustness can arise without stabilizing selection for any particular phenotype \cite{Siegal2002}. They used a gene interaction network model to demonstrate that robustness is an emergent property of complex networks subject only to selection on the basis of their own functional stability. The same authors used a similar in silico evolution approach to show that most genes in developmental networks buffer phenotypic variation, suggesting that robustness is an emergent byproduct of network architecture \cite{Bergman2003}. More broadly, additional computational studies have questioned the notion that global properties of biological networks are a consequence of adaptive evolution \cite{Lynch2007,Wagner2003}. These results suggest that local properties of genetic circuits can drive the emergence of macroscopic features, such as those that ensure robust development.

One intuitive strategy that cells could use to guarantee reliable cell fate decisions is the incorporation of functional redundancy \cite{Hartman2001,McAdams1999}. Indeed, when a gene or its products are absent or fail to perform their functions, the simplest contingency is to have others ready to compensate. Nowak et al. showed that genetic redundancy is particularly evolutionarily stable in developmental systems \cite{Nowak1997}. Using a simple binary outcome simulation procedure, the authors demonstrated that increasing the probability with which each gene in a functionally overlapping pair fails to carry out its function places increasing selection pressure upon both redundant genes. They reasoned that the resultant increase in evolutionary stability explains the prevalence of redundancy in developmental systems, where failure rates are high and erroneous cellular decisions propagate as deleterious phenotypes. Their suggestion is consistent with the extensive functional redundancy documented in developmental systems throughout the literature \cite{Kitano2004}. Redundancy may arise through gene duplication or convergent evolution. A. Wagner argued against the former by associating gene sequence and expression data in yeast, showing that the suppression of phenotypic severity in response to loss-of-function mutations does not correlate with the presence of functionally similar genes elsewhere in the genome \cite{Wagner2000}. He concluded that functional redundancy is not the primary cause of robustness to genetic variation, instead favoring epistatic interactions between otherwise unrelated genes. Combined, these results indicate that the prevalence of functional redundancy in GRNs may be attributed to as-of-yet unknown mechanisms.
 
\section{Overview of the current work}

Cell fate decision mechanisms are notoriously difficult to quantitatively characterize. Chapter \ref{ch:clones} introduces a computational framework for automated quantitative analysis of genetic mosaics; a class of experiments designed to probe gene expression \textit{in vivo}. The framework combines computer vision and statistics to measure protein levels in individual cells and predict their respective genotypes. It thereby facilitates systematic quantitative comparison of expression levels between cells subject to control and perturbation conditions. The accompanying software eliminates each of the labor-intensive steps of a quantitative workflow, and will hopefully lower the barrier to adoption of quantitative analysis strategies. 

Chapter \ref{ch:ratio} explores a novel cell fate decision mechanism underlying photoreceptor specification in the developing fruit fly eye. Computer vision and statistical modeling techniques are used to extract quantitative measurements of transcription factor dynamics from a wealth of confocal microscopy data. These data show that differentiation is driven by dynamic changes in the ratio between two transcription factors, and is agnostic to changes in their absolute concentrations as long as the ratio remains constant. A general model based on the statistical physics of transcription factor DNA binding shows that this phenomenon is a natural consequence of competition between the two transcription factors for common binding sites. The findings add a new dimension to our understanding of how transcription factors coordinate cell fate decisions, and showcase the importance of both quantitative and dynamic measurements for characterizing developmental systems. 

Chapter \ref{ch:metabolism} proposes a novel theory to explain why the regulatory networks that coordinate cell fate decisions often contain several repressors tasked with attenuating expression of a single target gene. The theory posits that auxiliary negative regulators enable development to proceed more quickly by mitigating erroneous cell fate decisions when cells are rapidly metabolizing. It is supported by a robust collection of qualitative experiments showing that a broad variety of repressor loss-of-function phenotypes are reversed when biosynthesis rates are artificially slowed. A quantitative modeling framework is used survey a hypothesized mechanistic origin of this effect. Namely, that auxiliary repressors help avert erroneous decisions by expanding cells capacity to buffer excess protein expression. Quantitative measurements of transcription factor activity serve to validate the predictions made by the modeling framework. As shorter developmental times confer a selective advantage upon organisms, these findings may represent a novel evolutionary driving force for increased robustness of cell fate decisions.

Beyond their insights into the mechanics of cell fate decisions, these efforts have spawned several computational tools that may prove valuable to the broader community (see \ref{appendix:software}). All such resources have been made freely available, and their continued development should help contribute toward the sustained prevalence of quantitative analysis of cell fate decisions.
